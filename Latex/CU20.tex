\begin{UseCase}{CU20}{Agregar Deporte}{
		\noindent Esta caso de uso servirá para que el Jefe de Fomento Deportivo pueda agregar Actividades Deportivas, que posteriormente serán llevadas a la práctica en las distinas Unidades Académicas.
		Para ello el Jefe de Fomento Deportivo dará click en el botón \IUbutton{ + Agregar } ubicado en la \IUref{}{Pantalla de Deportes \ref{deportes}}.	
	} \label{CU20_evento}

		\UCitem{Versión}{0.1}
		\UCitem{Autor}{Rosales González Carlos Andrés}
		\UCitem{Supervisa}{Mendoza García Bruno Alejandro}
		\UCitem{Actor}{Jefe de Fomento Deportivo}
		\UCitem{Propósito}{Agregar los datos de pruebas en la página.}
        \UCitem{Precondiciones}{
        \begin{itemize}
            \item Iniciar sesión.	
        \end{itemize}}
        \UCitem{Postcondiciones}{Se muestra la pantalla Agregar Deporte}
		\UCitem{Entradas}{
        \begin{itemize}
        	\item Nombre del deporte
        \end{itemize}}
		\UCitem{Origen}{Pantalla, Teclado}
		\UCitem{Salidas}{
		\begin{itemize}
		    \item Datos registrados.
		    \item Página de deportes.
		\end{itemize}}
		\UCitem{Destino}{Pantalla Agregar Deporte}
		\UCitem{Errores}{
        	\begin{itemize}
			    \item Datos faltantes.
			    \item Campo vacío.
            \end{itemize}
       }
		\UCitem{Observaciones}{}
		\end{UseCase}
	
    \begin{UCtrayectoria}{Principal}
    \UCpaso[\UCactor] Oprime el botón \IUbutton{ Agregar } en la \IUref{}{Pantalla Deportes \ref{deportes}}.
    \UCpaso Muestra la \IUref{}{Pantalla Editar un Agregar un Deporte \ref{agregadeporte}}. 
    \UCpaso[\UCactor] Lllena los campos. \label{CU20_regresar}  
    \UCpaso[\UCactor] Presiona el botón \IUbutton{ Registrar }.
    \UCpaso Comprueba que los campos no estén vacíos. \Trayref{A}
    \UCpaso Obtiene los valores ingresados
    \UCpaso Válida campos. \Trayref{B}
    \UCpaso Muestra mensaje de confirmación de registro.
    \UCpaso Muestra la \IUref{}{Pantalla Deportes. \ref{deportes}}.
\end{UCtrayectoria}

\begin{UCtrayectoriaA}{A}{Campo(s) vacios}
	\UCpaso Muestra mensaje “Campos Necesario".
	\UCpaso Continua en el paso \ref{CU20_regresar} del \UCref{CU20}.
\end{UCtrayectoriaA}

\begin{UCtrayectoriaA}{B}{Válida campos}
	\UCpaso Muestra el mensaje "Datos incorrectos".
	\UCpaso Continua en el paso \ref{CU20_regresar} del \UCref{CU20}.
\end{UCtrayectoriaA}


	


