\begin{UseCase}{CU2}{Inscribir a un evento interpolitécnico}{
		\noindent Este caso de uso permite que el actor alumno, pueda registrarse en el evento interpolitécnico deportivo de su interés. Deberá llenar un formulario donde se solicitan datos del alumno como: Grupo, NSS (Número de Seguro Social), correo electrónico, Delegación/Municipio, así como el seleccionar el deporte en el que desea participar.\\
        Para poder inscribirse, deberá primero validar su estatus académico (Inscrito/No inscrito), para ello debe ingresar su boleta, contraseña y el captcha como se muestra en la pantalla \IUref{p15InscripcionInterpolitecnico1}{Pantalla Inscribir interpolitécnico 1.}, da click en el botón \IUbutton { Verificar } si cumple con el requisito, continua el proceso, en caso contrario no podrá inscribirse en algun evento interpolitécnico deportivo.\\
        El siguiente paso es la verificación de datos como se muestra en la pantalla \IUref{p15InscripcionInterpolitecnico2}{Pantalla Inscribir interpolitécnico 2.}, si los datos son correctos da click en el botón \IUbutton{ Aceptar }\\
        Si los datos son correctos, da click en el botón \IUbutton{ Aceptar }, a continuación se muestra la pantalla \IUref{p15InscripcionInterpolitecnico3}{Pantalla Inscribir interpolitécnico 3.} donde llenará los datos corresponidentes al evento deportivo. Una vez que se llenen todos los campos, el alumno da click en el botón \IUbutton{ Inscribir }.\\ 
        Al final se mostrará un mensaje de confirmación de la inscripción.
	}
		\UCitem{Versión}{0.1}
		\UCitem{Autor}{Rosales González Carlos Andrés}
		\UCitem{Supervisa}{Mendoza García Bruno Alejandro}
		\UCitem{Actor}{Alumno}
		\UCitem{Propósito}{Poder participar en un evento deportivo.}
        \UCitem{Precondiciones}{
        \begin{itemize}
            \item Iniciar Sesión
            \item Ser un alumno pertenenciente al IPN
            \item Validar estatus académico
        \end{itemize}}
        \UCitem{Postcondiciones}{Persitencia de dat}
		\UCitem{Entradas}{
        \begin{itemize}
        	\item Boleta, contraseña y captcha
        	\item Grupo, Escuela, Carrera
        	\item Nombre, Apellido, Sexo
        	\item Curp, Fecha de nacimiento, Lugar
        	\item NSS, Correo electrónico, Delegación
        	\item Deporte, Sub-division, Prueba, Fecha del evento
        \end{itemize}}
		\UCitem{Origen}{Teclado}
		\UCitem{Salidas}{
		\begin{itemize}
		    \item Confirmación de la inscripción al evento
		\end{itemize}}
		\UCitem{Destino}{Pantalla principal}
		\UCitem{Errores}{
        	\begin{itemize}
            	\item EL alumno no se encuentra inscrito en el periodo actual.
            	\item Completa todos los campos
            \end{itemize}
       }
		\UCitem{Observaciones}{}
		\end{UseCase}
    \begin{UCtrayectoria}{Principal}
    \UCpaso[\UCactor] Oprime el botón \IUbutton { Inscribir Interpolitécnico } de la pantalla \IUref{p13Iniciopaticipante}{Pantalla principal del alumno.}.\label{CU3_inicio}
    \UCpaso Muestra la pantalla \IUref{p15InscripcionInterpolitecnico1}{Pantalla Inscribir interpolitécnico 1.}\label{CU3_regresa}
    \UCpaso[\UCactor] Ingresa boleta, contraseña y captcha.
    \UCpaso[\UCactor] Da click en el botón \IUbutton { Verificar }
    \UCpaso Envia los datos mediante el crawler a la página del SAES.
    \UCpaso Verifica si hay acceso al SAES. \Trayref{A} \Trayref{B}
    \UCpaso Muestra la pantalla \IUref{p15InscripcionInterpolitecnico2}{Pantalla Inscribir interpolitécnico 2.}.
    \UCpaso Muestra los datos personales del alumno.
	\UCpaso[\UCactor] Da click en el botón \IUbutton {Aceptar}. \Trayref{C}
	\UCpaso Busca los deportes y subdivisiones asociados a la unidad académica del alumno que esta haciendo la solicitud.
	\UCpaso Muestra la pantalla \IUref{p15InscripcionInterpolitecnico3}{Pantalla Inscribir interpolitécnico 3.}. \label{CU3_deporte}
	\UCpaso[\UCactor] Llena los campos solicitados. \Trayref{D}
    \UCpaso Confirma registro en una ventana emergente.
    \UCpaso Carga la pantalla Principal.
    \end{UCtrayectoria}
    
	\begin{UCtrayectoriaA}{A}{El alumno debe de estar inscrito para continuar.}
		\UCpaso Muestra el mensaje. “El alumno no esta inscrito en el periodo actual”
   		\UCpaso Continua en el paso \ref{CU3_regresa} del \UCref{CU3}.
	\end{UCtrayectoriaA}
	
	\begin{UCtrayectoriaA}{B}{El alumno cancela el proceso de Inscribir interpolitécnico}
		\UCpaso[\UCactor] Da click en el botón \IUbutton { Cancelar } de la pantalla \IUref{p15InscripcionInterpolitecnico1}{Pantalla Inscribir interpolitécnico 1.}
		\UCpaso  Continua en el paso \ref{CU3_inicio} del \UCref{CU3}.
	\end{UCtrayectoriaA}

	\begin{UCtrayectoriaA}{C}{Los datos del alumno no coinciden}
		\UCpaso[\UCactor] Da click en el botón \IUbutton { Cancelar } de la pantalla \IUref{p15InscripcionInterpolitecnico2}{Pantalla Inscribir interpolitécnico 2.}
		\UCpaso Continua en el paso \ref{CU3_inicio} del \UCref{CU3}.
	\end{UCtrayectoriaA}
	
	\begin{UCtrayectoriaA}{D}{El alumno no completa los campos requeridos}
		\UCpaso Muestra el mensaje "Debes llenar todos los campos solicitados". \ref{CU3_deporte}
		\UCpaso Continua en el paso \ref{CU3_inicio} del \UCref{CU3}.
	\end{UCtrayectoriaA}