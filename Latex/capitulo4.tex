\chapter{Prototipo para el Registro a Interpolit\'ecnicos Deportivos (RIDESCOM)}
	
	
	%=========================================================
	%                                                         Prototipo RIDESCOM
	%=========================================================
	\section{Prototipo para el Registro a Interpolit\'ecnicos Deportivos (RIDESCOM)}
	
	%=========================================================
	%                                             Requisitos
	%=========================================================
	\section{Reglas de Negocio}
	\noindent La aplicación RIDESCOM permitirá a los alumnos que estén inscritos en el periodo actual, inscribirse en un evento interpolitécnico, consultar los eventos registrados, consultar calendario de eventos. 
	El sistema cuenta con una interfaz que nos permite visualizar las diferentes opciones ofrecidas por el mismo. Para acceder al anterior el usuario debe de tener un usuario y contraseña. Dicho usuario y contraseña, será con el que ingresa al sistema del SAES. 
	Así mismo, habrá “usuarios encargados”, quienes se encargan de controlar y supervisar el uso de éste ante pequeñas porciones de estudiantes (grupos de alumnos). Ahora dichos encargados hay dos tipos el Jefe de Fomento Deportivo, a este se le permitirá crear eventos deportivos, registrar eventos deportivos y restablecer contraseña a los perfiles de los coordinadores. El segundo tipo de encargado es el Coordinador de una Unidad Académica, este podrá registrar los resultados de los participantes en el sistema para poder ser visualizados en la vista principal, podrá generar un reporte de los alumnos registrados durante el periodo que se haya especificado y registrar a los entrenadores que laboran dentro de la Unidad Académica a la que atienda.
	
	
	%=========================================================
	%                                             Requisitos
	%=========================================================
	\section{Reglas del Sistema}
	\noindent El alumno que desee participar en un evento interpolitécnico hará uso de la aplicación web RIDESCOM. Como primer punto deberá iniciar sesión en la misma, ingresando el usuario y contraseña con el que entra a sistema SAES (Sistema de Administración Escolar). Si los datos ingresados son correctos, se le dará acceso a la aplicación RIDESCOM, en caso contrario no se le dará acceso para poder registrarse en un evento. Sin embargo, podrá seguir visualizando datos generales, como lo es el calendario de eventos, resultados de eventos y los eventos que se practican dentro de la unidad académica.
	El Jefe de Fomento Deportivo podrá dar de alta a un Coordinador de alguna Unidad Académica. Para dar de alta un evento deportivo deberá llenar todos los campos requeridos, tendrá la opción de agregar una descripción si así lo desea. 
	El Coordinador de la Unidad Académica registrará a los entrenadores de las actividades deportivas deberá llenar los campos requeridos para poder concluir el registro. En caso de que exista un entrenador ya haya sido registrado, la aplicación le notificará. Una vez concluido los eventos deportivos, este registrará los resultados obtenidos por los participantes para que puedan ser vistos por la comunidad en general.
	
	
	%=========================================================
	%                                            Casos de Uso
	%=========================================================
	\section{Casos de Uso}
	\begin{UseCase}{CU1}{Iniciar Sesión Jefe de Fomento Deportivo}{
		\noindent Esta caso de uso servirá para que el Jefe de Fomento Deportivo pueda ingresar a la página web, poder identificar al usuario y así mostrar las vistas que tienen asginada. \\
    	Para poder iniciar sesión el actor deberá oprimir el botón \IUbutton{ Iniciar Sesión } ubicado en la pantalla \ref{inicioJFDycoord}. Ingresará su número de boleta, contraseña el cual usa para ingresar al SAES (Sistema de Administración Escolar), y el captcha, como se muestra en la pantalla... Si los datos que ingresa no coinciden se le mostrará un mensaje.
        Una vez que incie sesión se le mostrará la pantalla principal.
        \MSGref{MSG1}{Faltan campos por completar}.
        
	} \label{CU1_Iniciarsesion}
		\UCitem{Versión}{0.1}
		\UCitem{Autor}{Rosales González Carlos Andrés}
		\UCitem{Supervisa}{Mendoza García Bruno Alejandro}
		\UCitem{Actor}{Jefe de Fomento Deportivo}
		\UCitem{Propósito}{Tener control de las personas registradas.}
        \UCitem{Precondiciones}{
        \begin{itemize}
            \item Contar con una cuenta.
            \item Contar con la contraseña.
        \end{itemize}}
        \UCitem{Postcondiciones}{Se muestra la pantalla principal}
		\UCitem{Entradas}{
        \begin{itemize}
        	\item Usuario. 
        	\item Contraseña
        \end{itemize}}
		\UCitem{Origen}{Pantalla, Teclado}
		\UCitem{Salidas}{
		\begin{itemize}
		    \item Acceso a la página principal del Jefe de Fomento Deportivo
		\end{itemize}}
		\UCitem{Destino}{Pantalla}
		\UCitem{Errores}{
        	\begin{itemize}
        	    \item Los campos están vacíos.
            	\item Usuario y/o contraseña incorrecta.
            \end{itemize}
       }
		\UCitem{Observaciones}{}
		\end{UseCase}
	\newpage
	
    \begin{UCtrayectoria}{Principal}
    \UCpaso[\UCactor] Ingresa a la página RIDESCOM.
    \UCpaso Muestra la pantalla \IUref{}{Pantalla de Inicio de Sesión \ref{inicioJFDycoord}}.
    \UCpaso[\UCactor] Oprime el botón \IUbutton{ JFD o Coordinador } que esta en la \IUref{}{Pantalla de Inicio de Sesión \ref{inicioJFDycoord}}.
    \UCpaso Muestra la \IUref{}{Pantalla de Inicio de Sesión \ref{inicioJFDycoord}}
	\UCpaso[\UCactor] Introduce Usuario y contraseña. \label{CU1_regresar} 
    \UCpaso[\UCactor] Presiona el botón \IUbutton{ Ingresar }.
    \UCpaso Comprueba que los campos no estén vacíos. \Trayref{A}
    \UCpaso Obtiene los valores ingresados
    \UCpaso Válida campos. \Trayref{B}
    \UCpaso Muestra la \IUref{}{Pantalla principal del Jefe de Fomento Deportivo. \ref{principalJFD}}
    \end{UCtrayectoria}
    
    \begin{UCtrayectoriaA}{A}{Campo(s) vacios}
    	\UCpaso muestra mensaje “CamposNecesario".
    	\UCpaso Continua en el paso \ref{CU1_regresar} del \UCref{CU1}.
    \end{UCtrayectoriaA}

	\begin{UCtrayectoriaA}{B}{Boleta y/o contraseña erróneo}
		\UCpaso muestra mensaje “El usuario y/o contraseña que se ingresó son erróneos”. Mensaje .
   		\UCpaso Continua en el paso \ref{CU1_regresar} del \UCref{CU1}.
	\end{UCtrayectoriaA}

	



	\begin{UseCase}{CU1.1}{Inicio Sesión}{
		Servirá para que el alumno pueda ingresar a la aplicación y así poder inscribirse en algún evento de su interés o consultar los eventos a los que ya se ha registrado previamente. \\
        En caso de que el alumno ingrese una boleta la cual no a sido registrada, se mostrará un mensaje el cual le indique que la boleta que ingreso no existe. Mensaje . De igual manera, si la contraseña es diferente a la que se registro aparecerá un mensaje que le indique que la contraseña no coincide. Mensaje .
	}
		\UCitem{Versión}{0.1}
		\UCitem{Autor}{Rosales González Carlos Andrés}
		\UCitem{Supervisa}{Mendoza García Bruno Alejandro}
		\UCitem{Actor}{Alumno}
		\UCitem{Propósito}{Tener control de las personas registradas.}
        \UCitem{Precondiciones}{
        \begin{itemize}
            \item Haberse registrado
            \item Perfil valido por el coordinador
        \end{itemize}}
        \UCitem{Postcondiciones}{Ninguna}
		\UCitem{Entradas}{
        \begin{itemize}
        	\item Número de boleta 
        	\item Contraseña
        \end{itemize}}
		\UCitem{Origen}{Pantalla, Teclado}
		\UCitem{Salidas}{
		\begin{itemize}
		    \item Acceso a la página principal del alumno
		\end{itemize}}
		\UCitem{Destino}{Pantalla}
		\UCitem{Errores}{
        	\begin{itemize}
        	    \item Los campos están vacíos.
            	\item No existe la boleta. Mensaje .
            	\item Contraseña incorrecta. Mensaje .
            \end{itemize}
       }
		\UCitem{Observaciones}{}
		\end{UseCase}
    \begin{UCtrayectoria}{Principal}
    \UCpaso[\UCactor] Oprime el botón Iniciar Sesión en la pantalla.
    \UCpaso Muestra la pantalla.
	\UCpaso[\UCactor] Introduce Boleta y contraseña. 
    \UCpaso[\UCactor] Presiona el botón Ingresar.
    \UCpaso Comprueba que los campos no estén vacíos. \Trayref{A} \Trayref{B} \Trayref{C}
    \UCpaso Obtiene los valores ingresados
    \UCpaso Válida campos. 
    \UCpaso Muestra la pantalla .
    \end{UCtrayectoria}
    
	\begin{UCtrayectoriaA}{A}{No hay dato insertado en el campo solicitado}
		\UCpaso muestra mensaje “Error: Los campos están vacíos por favor asegúrese de poner lo que se pide”. Mensaje .
		\UCpaso Regresa al paso 3 de la Trayectoria Principal.
	\end{UCtrayectoriaA}
	
	\begin{UCtrayectoriaA}{B}{}
		\UCpaso muestra mensaje “No existe boleta ingresada”. Mensaje .
		\UCpaso Regresa al paso 2 de la trayectoria principal.
	\end{UCtrayectoriaA}
	
	\begin{UCtrayectoriaA}{C}{}
		\UCpaso muestra mensaje “Contraseña incorrecta”. Mensaje .
		\UCpaso Regresa al paso 2 de la trayectoria principal.
	\end{UCtrayectoriaA}
	\begin{UseCase}{CU2}{Inscripción a un evento interpolitécnico}{
		Servirá para que el alumno pueda registrarse en el evento de su interés, dentro de este encontrará un formulario donde se le solicita: Grupo, NSS (Número de Seguro Social), correo electrónico, Delegación/Municipio, así como el seleccionar el deporte en el que desea participar.\\
        Para poder inscribirse, deberá primero validar su estatus (Inscrito/No inscrito), en caso de que el alumno no valide su estatus el botón Registrar estará deshabilitado y no podrá registrarse.\\
        Si el alumno valida su estatus académico y este no cumple con los requisitos, de igual manera el botón Registrar estará deshabilitado.
	}
		\UCitem{Versión}{0.1}
		\UCitem{Autor}{Rosales González Carlos Andrés}
		\UCitem{Supervisa}{Mendoza García Bruno Alejandro}
		\UCitem{Actor}{Alumno}
		\UCitem{Propósito}{Poder participar en un evento deportivo.}
        \UCitem{Precondiciones}{
        \begin{itemize}
            \item Haberse registrado en el sistema
            \item Iniciar Sesión
            \item Validar inscripción
        \end{itemize}}
        \UCitem{Postcondiciones}{Ninguna}
		\UCitem{Entradas}{
        \begin{itemize}
        	\item Boleta, Grupo, Escuela, Carrera
        	\item Nombre, Apellido, Sexo
        	\item Curp, Fecha de nacimiento, Lugar
        	\item NSS, Correo electrónico, Delegación
        	\item Deporte, Sub-division, Prueba, Fecha del evento
        \end{itemize}}
		\UCitem{Origen}{Pantalla}
		\UCitem{Salidas}{
		\begin{itemize}
		    \item Confirmación de la inscripción al evento
		\end{itemize}}
		\UCitem{Destino}{Pantalla principal}
		\UCitem{Errores}{
        	\begin{itemize}
            	\item EL alumno no se encuentra inscrito en el periodo actual.
            	\item Completa todos los campos
            \end{itemize}
       }
		\UCitem{Observaciones}{}
		\end{UseCase}
    \begin{UCtrayectoria}{Principal}
    \UCpaso[\UCactor] Oprime el botón Inscribir Interpolitécnico de la pantalla .
    \UCpaso Busca los eventos interpolitecnicos registrados, activos y en los que participa la escuela del alumno.
    \UCpaso Muestra la pantalla .
	\UCpaso[\UCactor] Selecciona el evento 
	\UCpaso Busca los deportes y subdivisiones asociados al evento seleccionado
	\UCpaso[\UCactor] Selecciona el deporte
    \UCpaso busca las pruebas asociadas al deporte seleccionado.
    \UCpaso[\UCactor] Ingresa los datos solicitados. \Trayref{A} \Trayref{B}
    \UCpaso Confirma registro en una ventana emergente.
    \UCpaso Carga la pantalla Principal.
    \end{UCtrayectoria}
    
	\begin{UCtrayectoriaA}{A}{}
		\UCpaso Muestra el mensaje. “El alumno no esta inscrito en el periodo actual”
		\UCpaso El botón para completar el registro permanecerá inhabilitado.

	\end{UCtrayectoriaA}
	
	\begin{UCtrayectoriaA}{B}{}
		\UCpaso Muestra el mensaje. “Completa todos los campos”
		\UCpaso Regresa al paso 3 de la Trayectoria Principal.

	\end{UCtrayectoriaA}
	\begin{UseCase}{CU4}{Recupera contraseña para el alumno}{
			\noindent Servirá para que el alumno pueda recuperar su contraseña, en caso de que esté la olvidará. Para ello deve dar click en el botón \IUbutton{ Recuperar Contraseña } que se encuentra en la \IUref{}{Pantalla de Inicio de Sesión \ref{loginalumno}}. 
		Será re-dirigido a la página del SAES de la Unidad Académica a la que pertenece para que de estea manera pueda solicitar su restablecimiento de contraseña.	
}
		\label{CU4_Recuperaalum}
	
	\UCitem{Versión}{0.1}
	\UCitem{Autor}{Rosales González Carlos Andrés}
	\UCitem{Supervisa}{Mendoza García Bruno Alejandro}
	\UCitem{Actor}{Alumno}
	\UCitem{Propósito}{Restablecer la contraseña.}
	\UCitem{Precondiciones}{Contar con una cuenta registrada.}
	\UCitem{Postcondiciones}{
		\begin{itemize}
			\item El alumno podrá restablecer su contraseña.
			\item Ingresar a la página.
	\end{itemize}}
	\UCitem{Entradas}{
		\begin{itemize}
			\item Correo electrónico
	\end{itemize}}
	\UCitem{Origen}{Pantalla, Teclado}
	\UCitem{Salidas}{Pantalla}
	\UCitem{Destino}{Login Alumno}
	\UCitem{Errores}{
		\begin{itemize}
			\item La boleta no es válida.
			\item Correo Inválido.
		\end{itemize}
	}
	\UCitem{Observaciones}{Ninguna}
\end{UseCase}
\begin{UCtrayectoria}{Principal}
	\UCpaso[\UCactor] Oprime el botón \IUbutton{ Recuperar } ubicado en la \IUref{}{Pantalla de Inicio de Sesión \ref{loginalumno}}.
	\UCpaso Es redirigido a la página del SAES de ESCOM.
	\UCpaso[\UCactor] Introduce su correo electrónico.
	\UCpaso[\UCactor] Presiona el botón \IUbutton{ Recuperar }.
	\UCpaso Envía la contraseña al correo registrado.
\end{UCtrayectoria}

	\begin{UseCase}{CU}{Validación de perfil}{
		Servirá para que el alumno que esté interesado en participar en algún evento interpolitécnico deportivo, cree una cuenta para posteriormente poder iniciar sesión y así, inscribirse en el evento de su interés. 
		Dicho registro lo encontrará dentro de la pantalla de Inicio en el apartado ‘Regístrate’, posteriormente deberá llenar los campos que se le solicitan, los cuales son: Boleta, Correo electrónico y una contraseña.
		El numero de Boleta consta de 10 caracteres numéricos, y en el correo solamente se aceptan los dominios más comunes (Gmail, Hotmail, Outlook).
		Una vez realizado, el alumno deberá acudir al Departamento de Actividades Deportivas de su Unidad Académica en un periodo no máximo a los 3 días a partir del día en el que se registró, para que el coordinador valide los datos que se ingresaron previamente. Para ello el coordinador deberá solicitar una identificación escolar vigente para corroborar dichos datos. }
		\label{CU_Validacionperfil}
	
	\UCitem{Versión}{0.1}
	\UCitem{Autor}{Rosales González Carlos Andrés}
	\UCitem{Supervisa}{Mendoza García Bruno Alejandro}
	\UCitem{Actor}{Alumno}
	\UCitem{Propósito}{Poder inscribirse en un evento interpolitécnico deportivo.}
	\UCitem{Precondiciones}{No estar registrado previamente}
	\UCitem{Postcondiciones}{
		\begin{itemize}
			\item El alumno podrá ingresar al sistema.
			\item Habrá un registro nuevo del alumno.
			\item Deberá acudir en un periodo no máximo a 3 días al Departamento de Actividades Deportivas de su Unidad Académica.
	\end{itemize}}
	\UCitem{Entradas}{
		\begin{itemize}
			\item Número de boleta 
			\item Contraseña
			\item Correo electrónico
	\end{itemize}}
	\UCitem{Origen}{Pantalla, Teclado}
	\UCitem{Salidas}{Pantalla}
	\UCitem{Destino}{Pantalla principal}
	\UCitem{Errores}{
		\begin{itemize}
			\item La boleta no es válida
			\item Dominio de correo invalido
		\end{itemize}
	}
	\UCitem{Observaciones}{Ninguna}
\end{UseCase}
\begin{UCtrayectoria}{Principal}
	\UCpaso[\UCactor] Oprime el \IUbutton{ Registrate  } ubicado en la pantalla Principal.
	%\UCpaso Muestra el mensaje {\bf MSG1-}``¿Está [{\em seguro}] de querer eliminar este registro.''.
	\UCpaso Se conecta al SAES y obtiene el CAPTCHA del login.
	\UCpaso Muestra la pantalla.
	\UCpaso[\UCactor] Introduce Boleta, Contraseña y correo electronico
	\UCpaso[\UCactor] Presiona el botón.
	\UCpaso Comprueba los campos obligatorios que no estén vacias.
	\UCpaso Inicia sesión en el SAES de la escuela usando la boleta, contraseña y captcha introducidos.
	\UCpaso verifica que el alumno está efectivamente inscrito \Trayref{A} \Trayref{B}
	\UCpaso Registra al alumno.
	\UCpaso Muestra el mensaje MSG1 “Registro de cuenta exitoso”.
	\UCpaso Muestra la pantalla .
\end{UCtrayectoria}

\begin{UCtrayectoriaA}{A}{Inserta algún otro carácter no correspondiente al “Número de Boleta” y presiona el botón ‘Registrar’}
	\UCpaso Muestra en la ventana el mensaje “Número de Boleta inválido”
	\UCpaso Regresa al paso 2 de la trayectoria principal.
\end{UCtrayectoriaA}

\begin{UCtrayectoriaA}{B}{Inserta algún otro carácter no correspondiente al “Dominio del correo electrónico” y presiona el botón ‘Registrar’}
	\UCpaso Muestra en la ventana el mensaje “Correo inválido, asegúrese que su correo sea de tipo Gmail, Hotmail o Outlook”
	\UCpaso Regresa al paso 2 de la trayectoria principal.
\end{UCtrayectoriaA}
	
	
	%=========================================================
	%                                                         Requisitos de interaccion con el usuario
	%=========================================================
	\section{Requisitos del usuario}
	\begin{table}[htbp]
		\begin{center}
			\begin{tabular}{|l|p{45mm}|p{45mm}|p{45mm}|l}
				\hline
				Id & Nombre & Descripción & Prioridad \\
				\hline 
				RF1 & Registro de eventos & En la aplicación web se podrán registrar, modificar, eliminar y consultar  en un formulario todos los datos para identificar un evento.
				& MEDIA \\ \hline
				RF2 & Registro de participantes & En la aplicación web se podrán registrar, modificar, eliminar y consultar  en un formulario los datos del participante & ALTA  \\ \hline
				RF3 & Vista al público & En una pantalla se mostrarán los participantes que estén registrados en la aplicación y ver sus resultados de competencia. & MEDIO \\ \hline
				RF4 & Conexión con red social FACEBOOK. & Gracias a los datos que identifican a un evento se podrá promover en la red social FACEBOOK mediante el uso de API.& ALTA \\ \hline
				RF5 & Realizar una interfaz para los participantes (alumnos). &Se creará un(una ventana)  sitio para los alumnos que quieran participar en algún evento deportivo(, haciendo su registro, consultar estatus). & MEDIA \\ \hline
				RF6 & Mostrar una tabla de estadísticas. & En una pantalla (vista)  se mostrará todas las áreas deportivas que participaron en el evento deportivo y  número de participantes. & ALTA \\ \hline
				RF7 & Registrar un coordinador & El coordinador que utilizará la aplicación web tendrá que ser registrado en la base de datos. & ALTA \\ \hline
				RF8 & Vista para el coordinador. &El coordinador tendrá una vista donde podrá dar de alta eventos, participantes y generar cédulas de inscripción. & MEDIA \\ \hline
				RF9  & Historial & Para que se tenga un monitoreo de participantes. & ALTA \\ \hline
			\end{tabular}
			\caption{Requerimientos del Usuario.}
			\label{tabla:sencilla}
		\end{center}
	\end{table}
	\pagebreak
	
	%=========================================================
	%                                                         Requisitos funcionales
	%=========================================================
	\section{Requisitos funcionales de la aplicacion web}
	
	\begin{table}[htbp]
		\begin{center}
			\begin{tabular}{|l|p{45mm}|p{45mm}|p{45mm}|l}
				\hline
				Id & Nombre & Descripción & Prioridad \\
				\hline 
				RF1 & Validación de datos de los participantes. & La aplicación contará con un mecanismo de comprobación de estado académico (inscrito). & ALTA \\ \hline
				RF2 & Historial de participante. & Para tener seguimiento del participante durante su trayectoria académica & MEDIA  \\ \hline
				RF3 & Comunicación con la red social FACEBOOK &Habrá comunicación con la red social FACEBOOK para la publicación de eventos registrados en la aplicación.  & MEDIO \\ \hline
				RF4 & Creación de perfiles. & Se podrá asignar un perfil a un usuario.& MEDIA \\ \hline
			\end{tabular}
			\pagebreak
			\caption{Requerimientos funcionales de la aplicación web.}
			\label{tabla:sencilla}
		\end{center}
	\end{table}
	
	
	%=========================================================
	%                                                         Requisitos de informacion
	%=========================================================
	\section{Requisitos no funcionales de la aplicacion web}
	
	\begin{table}[htbp]
		\begin{center}
			\begin{tabular}{|l|p{45mm}|p{45mm}|p{45mm}|l}
				\hline
				Id & Nombre & Descripción & Prioridad \\
				\hline 
				RF1 & Vista de consulta genera. & Comunidad ajena a los participantes podrán ver los resultados. & MEDIA \\ \hline
				RF2 & Lista de registros &El usuario podrá consultar sus registros realizados & MEDIA   \\ \hline
				RF3 & Recuperación de contraseña &El usuario participante podrá recuperar su contraseña. & MEDIO \\ \hline
			\end{tabular}
			\caption{Requerimientos no funcionales de la aplicación web.}
			\label{tabla:sencilla}
		\end{center}
	\end{table}
	
	
	%=========================================================
	%                                                         Reglas de Negocio del Sistema
	%=========================================================
	\section{Reglas de necocio de la aplicación}
	
	