\chapter{Marco te\'orico}
	%=========================================================
	%                                                         Marco teorico
	%=========================================================
	
	\noindent Dentro del Instituto Politécnico Nacional (IPN) se han creado eventos que fomentan la participación y competitividad de la comunidad, llamados interpolitécnicos. Estos involucran áreas tales como: actividades deportivas, culturales o académicas, dichos eventos son de participación gratuita y se realizan 2 veces al año entre todos los planteles académicos que constituyen al IPN, divididas en los niveles Medio Superior y Superior.  \cite{Reglas}\\
	\noindent El IPN cuenta con 26 actividades deportivas registradas, sin embargo, en las unidades académicas no se practican todas.\cite{Reglas}.
	\noindent La entidad dedicada a la coordinación de las actividades deportivas con las que cuenta el IIPN, es la Dirección de Desarrollo y Fomento Deportivo \cite{DDYFD}. Esta se encarga de la creación, administración y control de todas las actividades deportivas prácticadas dentro del IPN tales como: definir el área donde se practican y la asignación de presupuesto de cada actividad deportiva, llevar un registro de la cantidad de población que practica un deporte. \cite{Reglamento}.	
	\noindent A su vez coordinan la realización de los eventos Interpolitécnicos Deportivos del IPN siguiendo ‘el reglamento general liga interpolitécnica. \cite{Reglamento}, donde se explican los  procesos que realiza cada persona involucrada.\\
	
	\noindent El Coordinador de Área Deportiva de cada Unidad Académica es el responsable de supervisar los aspectos operativos y técnicos de todos los deportes que se practican dentro de la misma así mismo es el encargado de realizar el proceso de inscripción a un interpolitécnico para los alumnos que así lo deseen y para que puedan participar este deberá solicitar la documentación de inscripción (cédula de inscripción) individual o de sus equipos y entregarlos a los Coordinadores de cada Disciplina Deportiva en la Dirección de Desarrollo y Fomento Deportivo. \cite{Reglamento} \\
	\noindent  El alumno que desee participar en un evento interpolitécno deberá acudir con el coordinador de su unidad académica para comenzar el proceso de inscripción a un evento interpolitécnico. 
	\\El coordinador le solicitará una forma para comprobar su estatus académico, este puede variar dependiendo de los coordinadores de las distintas unidades académicas. A su vez el alumno llenará el formato de inscripción al evento de su interés, anexando una fotografía. 
	\\ Si se comprueba que el alumno está inscrito en el periodo actual en el que quiere participar, podrá continuar con el proceso, en caso contrario se negará la inscripción. \cite{Reglamento}
	\\ Al concluir con la comprobación de inscripción, se le notifica al alumno cual es el estatus de su solicitud. 
	
	\noindent Para más detalles puede consultarse en el apartado Anexos Apartado \ref{ProcesoInscripcionActual}
	\pagebreak
	
	
	
	%=========================================================
	%                                                         Analisis de factibilidad tecnica
	%=========================================================
	\section{An\'alisis de Entornos de Desarrollo Interactivo}
	Éste análisis tiene como objetivo describir las herramientas que pueden ocuparse para la realización de este proyecto mencionando sus objetivos de cada herramienta de trabajo seleccionada.
	
	%=========================================================
	%                                                         IDE
	%=========================================================
	\subsection{IDE}
	\begin{itemize}
		\item  Netbeans 
		\label{Herramientas}
		\newline
		Netbeans es un entorno integrado de desarrollo o IDE (IntegratedDevelopmentEnvironment), cone el que se puede realizar todas las tareas asociadas a la programación.
		\newline
		Simplifica alguna de las tareas que, sobretodo en proyectos grandes, son laboriosas. Ofrece la posibilidad de asistencia (parcialmente) en la escritura de código, aunque no nos libera de aprender el lenguaje de programación.
		Nos ayuda en la navegación de las clases predefinidas.
		Aunque puede ser costoso su aprendizaje, los beneficios superan las dificultades.
		
	\subsection{Framework}
		\item Spring MVC \\ 
		Spring Mvc es una alternativa de framework basado en el patrón modelo-vista-controlador, después de haber aprendido de errores de frameowrks como Jakarta Struts y otras alternativas.
		El framework tiene un conjunto de interfaces que después se implementan para proporcionar la funcionalidad correspondiente. Las interfaces están acopladas claramente al Servlet Api.\cite{spring}\\
		La clase DispatcherServlet está en el front controller y es responsable de delegar y coordinar el control entre varias interfaces en la fase de ejecución durante una petición Http.
		Las interfaces más importantes definidas en Spring Mvc, y sus responsabilidades, son las siguientes:
		\begin{itemize}
			\item HandlerMapping: permite manejar peticiones de entrada.
			\item HandlerAdapter: ejecución de objetos que permiten manejar las peticiones entrantes.
			\item Controller: está entre el modelo y la vista, y permite manejar peticiones entrantes y redirigirlas a la respuesta adecuada. 
			\item Vista: responsable de retornar una respuesta al cliente. 
			\item ViewResolver: selecciona una vista basada en un nombre lógico de la vista.
			\item HandlerInterceptor: intercepta las peticiones entrantes, es comparable pero no igual a los filtros de Servlet.
			\item LocaleResolver: resuelve y opcionalmente salva el locale de un usuario individual.
			\item MultipartResolver: facilita trabajar con ficheros de subida wrapping peticiones de entrada.
		\end{itemize}
		Cada interfaz de estrategia tiene su responsabilidad importante dentro del framework general. Estas abstracciones ofrecidas por estas interfaces son potentes, pues permiten configurar un conjunto de interfaces juntas y ofrecen un conjunto en el top del Api de Servlet. Sin embargo, los desarrolladores y los vendedores son libres es escribir otras implementaciones. Spring MVC utiliza la interfaz java.util.Map como una abstracción del modelo cuando se espera que las llaves tenga valores de String. \\
		Cada testeo de las implementaciones de estas interfaces tiene otra ventaja importante dentro del alto nivel de abstracción ofrecido por Spring Mvc. Dispatcher Servlet está altamente acoplado con el contenedor de Inversión de Control de Spring para configurar las capas de web de las aplicaciones. Sin embargo, las aplicaciones web pueden utilizar otras partes del framework de Spring, incluyendo el contenedor, y decidir no usar Spring Mvc. \cite{MVC}\\
		Struts es un framework mucho más antiguo, por lo que Spring ha aprendido de la experiencia adquirida de usar Struts para no cometer los mismos errores. \cite{introspring}
		A continuación se enumeran un conjunto de ventajas:
		\begin{itemize}
			\item Spring MVC ofrece una división limpia entre Controllers, Models (JavaBeans) y Views. 
			\item Spring MVC es muy flexible, ya que implementa toda su estructura mediante interfaces, no como Struts que obliga a heredar de clases concretas tanto en sus Actions como en sus Forms. 
			\item Spring MVC provee interceptores y controllers que permiten interpretar y adaptar el comportamiento común en el manejo de múltiples requests.
			\item Los controllers de Spring MVC se configuran mediante IoC como los demás objetos, lo cual los hace fácilmente testeables e integrables con otros objetos que estén en el contexto de Spring, y por tanto sean manejables por éste. 
			\item Las partes de Spring MVC son más fácilmente testeables que las de Struts, debido a que evita la herencia de una clase de manera forzosa y una dependencia directa en el controller del servlet que despacha las peticiones. 
			\item No existen ActionForms, se enlaza directamente con los beans de negocio. 
			\item Struts obliga a extender la clase Action, mientras que Spring MVC no, aunque proporciona una serie de implementaciones de Controllers para que el usuario los utilice. Existe una gran variedad de Controladores.
			\item Spring tiene una interfaz bien definida para la capa de negocio. 
			\item Spring ofrece mejor integración con tecnologías distintas a JSP, como Velocity,XSLT,FreeMaker y XL. 
		\end{itemize}
		Las ventajas que se tiene usar Spring MVC 
		\begin{itemize}
			\item Se ha introduce un nombre es espacio MVC que simplifica la configuración.
			\item Se han insertado anotaciones adicionales como @CookieValue (permite relacionar un atributo de un método a una cookie) y @RequestMappings (asociar directamente en la clase la posibilidad de asociar una petición Url a un controlador específico).
			\item El tipo ConversionService es una alternativa más simple y robusta que los PropertyEditors de JavaBeans.
			\item Se soporta un formateo de números con el atributo @NumberFormat.
			\item Se soportan el formateo de fechas, calendarios y joda time con el atributo @DatetimeFormat, siempre que la librería Joda Time esté dentro del classpath.
			\item Se soporta la validación para entradas @Controller con la etiqueta @Valid, si se proporciona una implementación de la Jsr-303 en el classpath.
			\item Soporte para leer y escribir Xml, si Jaxb está dentro del classpath.
			\item Soporte para leer y escribir Json, si Jackson está dentor del classpath.
		\end{itemize}
		Para poder crear los controladores debemos de seguir los siguientes pasos:
		\begin{itemize}
			\item Definir una clase que implementa la interfaz del controlador.
			\item Insertar dicha clase como un objeto en el contexto de Spring.
			\item Asignar el nombre a una Uri que posteriormente será invocada por el usuario.
			\item Especificar la extensión de la uti en el DispatcherServlet de Spring, configurado en web.xml, para que se pueda buscar internamente.
		\end{itemize}
		
		El modelo son los datos con los que interactúa la vista y el controlador. En este caso, el modelo se pasa mediante un atributo de la clase ModelAndView, para posteriormente acceder al mismo desde la parte de la vista. \cite{introspring}
		
		Algunas recomendaciones para desarrollar usando Spring MVC
		Primero: determinar el Ide de desarrollo que se quiera utilizar. Existan varias alternativas: Netbeans, Eclipse, Intellij, Spring Tool Suite. Yo, por precio (gratis) y facilidad de uso (está orientado al desarrollo de Spring) recomiendo el uso de Spring Tool Suite. \\
		
		Segundo: crear el proyecto básico, y elegir nuestra herramienta de gestión de librerías. Para crear el proyecto, tenemos varias alternativas, desde configurar nosotros desde cero el servlet de spring, como utilizar plantillas ya existentes. También existe la posibilidad de descargar las librerías de la web de Spring o bien utilizar repositorios de Maven.\\
		
		Tercero: escoger el tipo de vista que se utilizará para el proyecto. En mi caso, la vista que suelo utilizar y con la que me siento más acostumbrado es Jsp.\\
		
		Cuarto: desarrollar la parte de negocio (controladores) y persistencia (seleccionando un framework como hibernate o jpa), y relacionar los controladores, con la vista y la persistencia.
		En principio, eso es todo.\\
		El framework fue lanzado inicialmente bajo Apache 2.0 License en junio de 2003. Esta licencia es una licencia de software libre creada por la Apache Software Foundation que requiere la conservación del aviso de copyright y disclaimer, pero no es una liencia copyleft, ya que no requiere la redistribución del código fuente cuando se distribuyen versiones modificadas. \cite{introspring} \\
		
		\item Facebook Developers 
		La Plataforma de Facebook es el conjunto de servicios, herramientas y productos proporcionados por el servicio de redes sociales Facebook para que los desarrolladores externos creen sus propias aplicaciones y servicios que acceden a datos en Facebook.\\
		
		La actual Plataforma de Facebook se lanzó en 2010. La plataforma ofrece un conjunto de interfaces y herramientas de programación que permiten a los desarrolladores integrarse con el "gráfico social" abierto de relaciones personales y otras cosas como canciones, lugares y páginas de Facebook. La asignación en facebook.com, sitios web externos y dispositivos pueden acceder al gráfico. \\
		
		Facebook lanzó la Plataforma de Facebook el 24 de mayo de 2007, proporcionando un marco para que los desarrolladores de software creen aplicaciones que interactúen con las funciones principales de Facebook. Se introdujo un lenguaje de marcado llamado Facebook Markup Language simultáneamente; se utiliza para personalizar el "aspecto y la sensación" de las aplicaciones que crean los desarrolladores. Usando la Plataforma, Facebook lanzó varias aplicaciones nuevas, incluyendo Regalos, permitiendo a los usuarios enviarse regalos virtuales entre sí, Marketplace, permitiendo a los usuarios publicar anuncios clasificados gratuitos, eventos de Facebook, brindando a los usuarios un método para informar a sus amigos sobre los próximos eventos, Video, Permitir a los usuarios compartir videos caseros entre ellos y juegos de redes sociales, donde los usuarios pueden usar sus conexiones con amigos para ayudarlos a avanzar en los juegos que están jugando. Muchos de los primeros juegos populares de redes sociales combinarían capacidades. Por ejemplo, uno de los primeros juegos en llegar al primer lugar de aplicación, (Lil) Green Patch, combina regalos virtuales con notificaciones de eventos a amigos y contribuciones a organizaciones benéficas a través de causas. \\
		
		Componentes de plataforma de alto nivel
		API de gráficos\\
		Graph API es el núcleo de la plataforma de Facebook, lo que permite a los desarrolladores leer y escribir datos en Facebook. Graph API presenta una vista simple y coherente del gráfico social de Facebook, que representa de manera uniforme los objetos en el gráfico (por ejemplo, personas, fotos, eventos y páginas) y las conexiones entre ellos (por ejemplo, relaciones de amigos, contenido compartido y etiquetas de fotos )\\
		
		Autenticación\\
		La autenticación de Facebook permite que las aplicaciones de los desarrolladores interactúen con la API Graph en nombre de los usuarios de Facebook, y proporciona un mecanismo de inicio de sesión único en aplicaciones web, móviles y de escritorio.\\
		
		Complementos sociales\\
		Los complementos sociales, incluidos el botón Me gusta, las recomendaciones y el feed de actividades, permiten a los desarrolladores proporcionar experiencias sociales a sus usuarios con solo unas pocas líneas de HTML. Todos los complementos sociales son extensiones de Facebook y están diseñados para que no se compartan datos de los usuarios con los sitios en los que aparecen. Por otro lado, los complementos sociales permiten a Facebook rastrear los hábitos de navegación de sus usuarios a través de cualquier sitio que cuente con los complementos. Y los datos recopilados de los hábitos de navegación de los usuarios ayudan a los vendedores y anunciantes en Facebook a dirigirse a su audiencia.\\
		
		iframes\\
		Facebook usa iframes para permitir que los desarrolladores externos creen aplicaciones que están alojadas por separado de Facebook, pero operan dentro de una sesión de Facebook y se accede a través del perfil de un usuario. Dado que los iframes esencialmente anidan sitios web independientes dentro de una sesión de Facebook, su contenido es distinto del formato de Facebook. \\
		
		Originalmente, Facebook utilizaba el 'Lenguaje de marcado de Facebook (FBML)' para permitir a los desarrolladores de aplicaciones de Facebook personalizar el "aspecto y la sensación" de sus aplicaciones, hasta cierto punto. FBML es una especificación de cómo codificar contenido para que los servidores de Facebook puedan leerlo y publicarlo, lo cual es necesario en el feed específico de Facebook para que el sistema de Facebook pueda analizar el contenido y publicarlo como se especifica. Facebook almacena en caché el FBML establecido por cualquier aplicación hasta que una llamada API posterior lo reemplace. Facebook también ofrece una biblioteca especializada de JavaScript de Facebook (FBJS).\\
		
		Facebook dejó de aceptar nuevas aplicaciones FBML el 18 de marzo de 2011, pero continuó admitiendo las pestañas y aplicaciones FBML existentes. Desde el 1 de enero de 2012, FBML ya no era compatible, y FBML ya no funcionaba a partir del 1 de junio de 2012. 
	\end{itemize}


	\subsection{Rastreo Web y API's}
	\begin{itemize}
		\item Web Crawler\newline
			Un Web Crawler (también llamado Web Spider) es un programa diseñado para explorar páginas Web en forma automática. La operación normal es que se le da al programa un grupo de direcciones iniciales, el crawler descarga estas direcciones, analiza las páginas y busca enlaces a páginas nuevas. Luego descarga estas páginas nuevas, analiza sus enlaces, y así sucesivamente.\cite{crawling} \\
			Los crawlers se pueden usar para varias cosas, lo más común es que se usen para:
				\begin{itemize}
					\item Crear el índice de una [article-1056.html máquina de búsqueda]. 
					\item Analizar los enlaces de un sitio para buscar links rotos. 
					\item Recolectar información de un cierto tipo, como precios de productos para armar un catálogo. \cite{craw}
				\end{itemize} 
			
			Un Web Crawler es un pequeño programa que recorre permanentemente el entramado de contenidos que conforman la Red. Su principal utilidad se la otorgan los buscadores al emplearlos para rastrear nuevas webs, de las que descargan automáticamente una copia que almacenan en un índice; una vez integradas en estas bases de datos las webs podrán ser rápidamente localizadas en la siguiente consulta efectuada por los usuarios, permitiendo mostrárselas entre la lista de resultados.
			Una araña web inicia su trabajo visitando un conjunto de direcciones predeterminadas, analiza las páginas, identifica los enlaces externos que éstas puedan incluir y los añade a la lista de direcciones a visitar, perpetuándose en su labor.	\cite{web} \\
			
			Ahora bien, un rastreador web, indexador web, indizador web o araña web es un programa informático que inspecciona las páginas del World Wide Web de forma metódica y automatizada.1 Uno de los usos más frecuentes que se les da consiste en crear una copia de todas las páginas web visitadas para su procesado posterior por un motor de búsqueda que indexa las páginas proporcionando un sistema de búsquedas rápido. Las arañas web suelen ser bots. \cite{araña} \\
			
			Web scraping es una técnica utilizada mediante programas de software para extraer información de sitios web. Usualmente, estos programas simulan la navegación de un humano en la World Wide Web ya sea utilizando el protocolo HTTP manualmente, o incrustando un navegador en una aplicación.  \\
			El web scraping está muy relacionado con la indexación de la web, la cual indexa la información de la web utilizando un robot y es una técnica universal adoptada por la mayoría de los motores de búsqueda. Sin embargo, el web scraping se enfoca más en la transformación de datos sin estructura en la web (como el formato HTML) en datos estructurados que pueden ser almacenados y analizados en una base de datos central, en una hoja de cálculo o en alguna otra fuente de almacenamiento. Alguno de los usos del web scraping son la comparación de precios en tiendas, la monitorización de datos relacionados con el clima de cierta región, la detección de cambios en sitios webs y la integración de datos en sitios webs.	\cite{araña} \\
			
			El web scraping es una técnica que sirve para extraer información de páginas web de forma automatizada. Si traducimos del inglés su significado vendría a significar algo así como “escarbar una web”. \\
			Aplicaciones y ejemplos:\\
			Su uso está muy claro: podemos aprovechar el web scraping para conseguir cantidades industriales de información (Big data) sin teclear una sola palabra. A través de los algoritmos de búsqueda podemos rastrear centenares de webs para extraer sólo aquella información que necesitamos.\\
			Para ello nos será muy útil dominar regex (regular expression) para delimitar las búsquedas o hacerlas más precisas y que el filtrado de la información sea mejor.\\
			Algunos ejemplos para los cuales vamos a necesitar el web scraping:
				\begin{itemize}
					\item Para marketing de contenidos: podemos diseñar un robot que haga un ‘scrapeo’ de datos concretos de una web y los podamos utilizar para generar nuestro propio contenido. Ejemplo: scrapear los datos estadísticos la web oficial de una liga de fútbol para generar nuestra propia base de datos.
					\item Para ganar visibilidad en redes sociales: podemos utilizar los datos de un scrapeo para interactuar a través de un robot con usuarios en redes sociales. Ejemplo: crear un bot en instagram que seleccione los links de cada foto y luego programar un comentario en cada entrada.
					\item Para controlar la imagen y la visibilidad de nuestra marca en internet: a través de un scrapeo podemos automatizar la posición por la que varios artículos de nuestra web se posicionan en Google o, por ejemplo, controlar la presencia del nombre de nuestra marca en determinados foros. Ejemplo: rastrear la posición en Google de todas las entradas de nuestro blog.
				\end{itemize}	

			
			Para poder desarrollar un web scraping de la mejor manera se debe considerar  dos vertientes muy diferenciadas del conocimiento web, ambas esenciales para tener perfil versátil en la red. Por un lugar debemos dominar la visualización de datos a nivel conceptual y por el otro debemos disponer de los conocimientos técnicos necesarios para lograr extraer con exactitud los datos con herramientas especializadas. \\
			Al fin y al cabo esto se resumirá en saber gestionar grandes cantidades de datos (big data). Debemos estar mínimamente familiarizados con la visualización de grandes cantidades de datos con tal de poder jerarquizar e interpretar los datos que extraigamos de una web. Y no solo para extraer los datos, también a la hora de plantear la estrategia de extracción debemos saber cuales van a ser los datos que vayamos a extraer con tal de poder darles un sentido informativo para el usuario \cite{scraping}.\\
			
			\textbf{Api's más utilizadas}
			\begin{itemize}
				\item \textbf{Jsoup}: Es una libreria de Java para trabajar con HTML real en el mundo real. Proporciona una API muy conveniente para extraer y manipular datos, utilizando lo mejor de DOM, CSS y métodos similares a jquery.\\
				
				Implementa la especificación HTML5 WHATWG y analiza HTML en el mismo DOM que los navegadores modernos.
				\begin{itemize}
					\item 
				\item Raspa y analiza HTML desde una URL, archivo o cadena
				\item Encuentra y extrae datos, usando DOM transversal o selectores de CSS
				manipular los elementos HTML, atributos y texto
				\item Limpia el contenido enviado por el usuario contra una lista blanca segura, para evitar ataques XSS
				\item HTML ordenado de salida
				\end{itemize}
				
				Jsoup está diseñado para tratar con todas las variedades de HTML que se encuentran en la naturaleza; de lo prístino y de la validación para invalidar la etiqueta soup; jsoup creará un árbol de análisis sensible.\\
				
				Jsoup es un proyecto de código abierto distribuido bajo la licencia liberal MIT. El código fuente está disponible en GitHub.\cite{jsoup}
				
				\item \textbf{Selenium}: Es un automatizador de navegadores. Principalmente, es para automatizar aplicaciones web con fines de prueba, pero ciertamente no se limita a eso. Las tareas de administración aburridas basadas en la web pueden (y deberían) ser automatizadas también.\\
				
				Selenium tiene el soporte de algunos de los proveedores más grandes de navegadores que han tomado (o están tomando) pasos para hacer de Selenium una parte nativa de su navegador. También es la tecnología central en muchas otras herramientas de automatización del navegador, API y marcos.\\
				
				Selenium provee una herramienta de grabar o reproducir para crear pruebas sin usar un lenguaje de scripting para pruebas (Selenium IDE). Incluye también un lenguaje específico de dominio para pruebas (Selanese) para escribir pruebas en un amplio número de lenguajes de programación populares incluyendo Java, C$\#$, Ruby, Groovy, Perl, Php y Python. Las pruebas pueden ejecutarse entonces usando la mayoría de los navegadores web modernos en diferentes sistemas operativos como Windows, Linux y OSX.
				
			\end{itemize}
		Como Api seleccionada para trepar páginas web Jsoup es apta para la realización del trabajo ya que nos permite analizar libremente el DOM de alguna página y es mucho más rápido que selenium, ya que no necesita molestarse con un DOM "vivo". Selenium siempre debe verificar si los manejadores de elementos siguen siendo válidos antes de realizar cualquier operación con ellos pero la sobrecarga es realmente notable cuando realiza un raspado serio. \\
			
	\end{itemize}
	
	
	%=========================================================
	%                                                         Sistema Gestor de BD
	%=========================================================
	%\section{Sistema Gestor de Base de Datos}
	%\noindent Realizando una investigación de los distintos gestores de Bases de Datos encontramos las siguientes caracteristicas:
	%\begin{table}[htbp]
	%	\begin{center}
	%		\begin{tabular}{|l|p{35mm}|p{35mm}|p{35mm}|l}
	%			\hline
	%			Caracter\'isticas & Oracle & MySQL & SQL Server \\
	%			\hline 
	%			Interfaz & GUI, SQL & SQL & GUI, SQL \\ \hline
	%			Lenguaje soportado & C, C++, C, Java, Ruby y Objective-C & C, C, C++, D, Java, Ruby y Objective C & Java, Ruby, Python, VB, .Net y PHP  \\ \hline
	%			Sistema Operativo & Windows, GNU/Linux, Solaris, OS-X & Windows, GNU/Linux, OS-X, FreeBSD, Solaris & Windows \\ \hline
	%			Licencia & Propietrio & Código Libre & Propietario \\ \hline
	%		\end{tabular}
	%		\caption{Tabla comparativa de lenguajes de BD.}
	%		\label{tabla:sencilla}
	%	\end{center}
	%\end{table}
	
	\section{iReport}
	\noindent IReport. Es una herramienta visual que sirve para generar ficheros XML (plantillas de informes) utilizando la herramienta de generación de informes JasperReport.\\
	\noindent Escrito en Java IReport provee a los usuarios de JasperReport una interfaz visual para construir reportes. También permite que los usuarios corrijan visualmente informes complejos con cartas, imágenes y subinformes. \\
	Bibliotecas gráficas que emplea \\
	\noindent Está además integrado con JFreeChart, una de la bibliotecas gráficas OpenSource más difundida para Java. Los datos para imprimir pueden ser recuperados por varios caminos incluso múltiples uniones JDBC, TableModels, JavaBeans, XML, etc. \\
	
	\textbf{Características de IReport}
	La lista siguiente describe algunas de las características importantes de IReport:
	\begin{itemize}
		\item 100 porciento escrito en Java y además OpenSource y gratuito.
		\item Maneja el 98 porciento de las etiquetas de JasperReport.
		\item Permite diseñar con sus propias herramientas: rectángulos, líneas, elipses, campos de los textfields, cartas, subreports (subreportes).
		\item Soporta internacionalización nativamente.
		\item Browser de la estructura del documento.
		\item Recopilador y exportador integrados .
		\item Soporta JDBC.
		\item Soporta JavaBeans como orígenes de datos (éstos deben implementar la interface JRDataSource).
		\item Incluye Wizard’s (asistentes) para crear automáticamente informes.
		\item Tiene asistentes para generar los subreportes.
		\item Tiene asistentes para las plantillas.
		\item Facilidad de instalación.
	\end{itemize}

	\textbf{Requerimientos de instalación}
	\begin{itemize}
		\item Sun JDK 1.4 (SDK) o superior.
		\item Acrobat 5.0 no es requerido, pero es fuertemente recomendado.
		\item Si se desea conectar con una base de datos, se debe proporcionar el DriverJDBC correspondiente.
		\item Usar la versión IReport-0.5.1 o superior.
	\end{itemize}

	\textbf{Librerias que utiliza}
		\begin{itemize}
			\item jasperreports-1.0.1.jar
			\item commons-digester.jar
			\item commons-beanutils.jar
			\item commons-collections.jar
			\item commons-logging.jar
			\item itext-1.02b.jar
			\item poi-2.0-final-20040126.jar
		\end{itemize}
	
	
	%=========================================================
	%              Sistema Gestor de BD del lado del servidor
	%=========================================================
	\section{Sistema Gestor de Base de Datos del lado del servidor}
	\noindent Para persistir la información que se genere a través de la aplicación y que esté disponible la mayor parte del tiempo por los usuarios es necesario contar con un SGBD en el servidor para que los datos están centralizados y se puedan agregar, actualizar, consultar y eliminar los datos generados por los usuarios. Comparando los SGBD, ver la tabla 2, más populares encontramos que la opción más confiable será utilizar MySQL debido a que es de código abierto, gratuito, con soporte técnico y abundante documentación. 
	
	\begin{table}[htbp]
		\begin{center}
			\begin{tabular}{|l|p{35mm}|p{35mm}|p{35mm}|l}
				\hline
				Caracter\'isticas & Oracle & MySQL & SQL Server \\
				\hline 
				Interfaz & GUI, SQL & SQL & GUI, SQL \\ \hline
				Lenguaje soportado & C, C++, C, Java, Ruby y Objective-C & C, C, C++, D, Java, Ruby y Objective C & Java, Ruby, Python, VB, .Net y PHP  \\ \hline
				Sistema Operativo & Windows, GNU/Linux, Solaris, OS-X & Windows, GNU/Linux, OS-X, FreeBSD, Solaris & Windows \\ \hline
				Licencia & Propietrio & Código Libre & Propietario \\ \hline
			\end{tabular}
			\caption{Tabla comparativa de lenguajes de BD.}
			\label{tabla:sencilla}
		\end{center}
	\end{table}
	\pagebreak
	
	%=========================================================
	%                                                         Servidor
	%=========================================================

	\section{Servidor}
	\noindent Para establecer las caracteristicas de nuestro servidor debemos conocer parte de sus componentes y que operabilidad tendrá dedicada.
	Empezando con los procesadores dedicados para equipos de servidores, se recomienda usar Intel Xeon (cualquier versión), por ejemplo: los servidores con Intel Xeon E5 v3 pueden tener hasta 36 núcleos y se encuentran entre los pocos procesadores que son compatibles con la nueva versión DDR4 de RAM, que utiliza un consumo de energía muy reducido y brinda una excelente velocidad de transferencia de datos.\cite{serv}
	
	Tomando en cuenta las recomendaciones para armar un buen servidor se tomaron algunas caracteristicas de componentes que pueden ser accesibles para montar el proyecto. \cite{serv}\cite{servi}
	\begin{table}[htbp]
		\begin{center}
			\begin{tabular}{|l|l|}
				\hline
				\multicolumn{2}{|c|}{Servidor Ideal} \\
				\hline
				Componentes & \\
				\hline
				\multicolumn{2}{|c|}{Hardware} \\
				\hline
				Procesador & Intel Xeon ó AMD Opteron\\
				\hline
				Plataforma & 64-Bits\\
				\hline
				Memoria RAM & 8 GB\\
				\hline
				Disco Duro & 500 GB\\
				\hline
				Arreglo de Discos Duros & Ninguno\\
				\hline
				Monitor & 800 x 600 16 bits Color o Superior\\
				\hline
				\multicolumn{2}{|c|}{Software} \\
				\hline
				Sistema Operativo & Windows \\
				\hline
				Base de Datos & MySQL 5.1 Community Server\\
				\hline
			\end{tabular}
			\caption{Tabla de especificaciones del servidor.}
		\end{center}
	\end{table}
Suponiendo el caso de que por día se hagan 1000 visitas diarias donde se tienen 6 páginas por visita que además por página consume 100kb.

En condiciones normales, una web envía más datos de los que recibe, por lo cual, hemos de contar con los datos enviados.
\\
Podremos calcular la transferencia de datos con la siguiente formula\\

(días por mes) x (visitas diarias) x (páginas por visita) x (volumen por página) x 1,25\\

En cuestión de la memoria RAM está esta pensada de manera que soporte las consultas realizadas por los usuarios en la aplicación, es bien dicho que entre más RAM el servidor tendrá mejor rendimiento \cite{servi}.

O bien Tomando como caso la capacidad del disco duro se considera que el almacenamiento no se saturará hasta pasado 1 año, puesto que el uso de memoria de almacenamiento de datos creados por la aplicación no serán concurrentes, con el proposito de almacenar la aplicación y los archivos de base de datos, cuya base de datos incrementa de acuerdo al número de usuarios que hagan uso de ella (unicamente los que estén inscritos, o usuarios coordinador).\\


