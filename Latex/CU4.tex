\begin{UseCase}{CU2.2}{Validación estatus}{
		\noindent Este caso de uso permite que el actor alumno comprube su estatus académico. Este proceso se realizará haciendo uso de la conexión con el SAES, para ello tendrá que dar click en el botón \IUbutton{ Inscribir Interpolitécnico } en la pantalla \IUref{p13Iniciopaticipante}{Pantalla principal del alumno.} 

	}
		\UCitem{Versión}{0.1}
		\UCitem{Autor}{Rosales González Carlos Andrés}
		\UCitem{Supervisa}{Mendoza García Bruno Alejandro}
		\UCitem{Actor}{Alumno/Sistema}
		\UCitem{Propósito}{Comprobar el estatus académico del alumno.}
        \UCitem{Precondiciones}{
        \begin{itemize}
            \item Haberse inciado sesión
        \end{itemize}}
        \UCitem{Postcondiciones}{Ninguna}
		\UCitem{Entradas}{
        \begin{itemize}
        	\item Boleta
        \end{itemize}}
		\UCitem{Origen}{Pantalla}
		\UCitem{Salidas}{
		\begin{itemize}
		    \item Inscrito, puedes participar
		    \item No inscrito, no puedes participar
		\end{itemize}}
		\UCitem{Destino}{Pantalla}
		\UCitem{Errores}{
        	\begin{itemize}
            	\item EL alumno no esta inscrito en el periodo actual y por tal motivo no puede participar en un evento deportivo.
            \end{itemize}
       }
		\UCitem{Observaciones}{}
		\end{UseCase}
    \begin{UCtrayectoria}{Principal}
    \UCpaso[\UCactor] Oprime Botón validar en la pantalla .
    \UCpaso Obtiene dato ingresado.
	\UCpaso Valida que el alumno esta inscrito 
    \UCpaso Muestra mensaje . \Trayref{A} \Trayref{B}
    \end{UCtrayectoria}
    
	\begin{UCtrayectoriaA}{A}{}
		\UCpaso Muestra el mensaje. “El alumno no esta inscrito en el periodo actual”
		\UCpaso El botón para completar el registro permanecerá inhabilitado.
		\UCpaso Regresa a trayectoria Principal 4.
	\end{UCtrayectoriaA}
	
	\begin{UCtrayectoriaA}{B}{}
		\UCpaso Muestra el mensaje. “El alumno esta inscrito en el periodo actual”
		\UCpaso El botón para completar el registro estará habilitado.
        \UCpaso Regresa a trayectoria Principal 4.
	\end{UCtrayectoriaA}