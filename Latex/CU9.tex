\begin{UseCase}{CU9}{Eliminar datos del Eventos Deportivos}{
		\noindent Esta caso de uso servirá para que el Jefe de Fomento Deportivo pueda eliminar un evento previamente registrado.
		Para ello el Jefe de Fomento Deportivo dará click en el botón \IUbutton{ Eliminar } ubicado en la  parte derecha inferior del recuadro correspondiente al evento. Como se muestra en la \IUref{}{Pantalla de Principal \ref{principalJFD}}.	
	} \label{CU9_evento}

		\UCitem{Versión}{0.1}
		\UCitem{Autor}{Rosales González Carlos Andrés}
		\UCitem{Supervisa}{Mendoza García Bruno Alejandro}
		\UCitem{Actor}{Jefe de Fomento Deportivo}
		\UCitem{Propósito}{Eliminar datos de eventos Interpolitécnicos Deportivos.}
        \UCitem{Precondiciones}{
        \begin{itemize}
            \item Iniciar sesión.
            \item Tener un evento registrado.
            \item Seleccionar un evento.
        \end{itemize}}
        \UCitem{Postcondiciones}{Se muestra la pantalla principal}
		\UCitem{Entradas}{
        \begin{itemize}
        	\item Nombre del evento 
        	\item Dirección
        	\item Sede
        	\item Punto de Referencia
        	\item Semestre
        	\item Fecha inicio de registro
        	\item Fecha fin de registro
        	\item Deporte
        	\item Comentarios
        \end{itemize}}
		\UCitem{Origen}{Pantalla, Teclado}
		\UCitem{Salidas}{
		\begin{itemize}
		    \item Evento eliminado
		    \item No se puede eliminar el evento.
		\end{itemize}}
		\UCitem{Destino}{Principal Jefe Fomento Deportivo}
		\UCitem{Errores}{
        	\begin{itemize}
        	    \item Los campos están vacíos.
            \end{itemize}
       }
		\UCitem{Observaciones}{}
		\end{UseCase}
	
    \begin{UCtrayectoria}{Principal}
    \UCpaso[\UCactor] Ingresa a la \IUref{}{Pantalla Editar un Evento Interpolitécnico Deportivo \ref{principalJFD}}.
    \UCpaso Muestra la \IUref{}{Pantalla Registrar un Evento Interpolitécnico Deportivo \ref{principalJFD}}.
    \UCpaso[\UCactor] Da click en el botón \IUbutton{ Eliminar }. \label{CU9_regresar} 
    \UCpaso Muestra mensaje para confirmar la acción. \Trayref{A}
    \UCpaso Muestra la \IUref{}{Pantalla principal del Jefe de Fomento Deportivo. \ref{principalJFD}}.
    \end{UCtrayectoria}
    
    \begin{UCtrayectoriaA}{A}{Error al eliminar el evento}
    	\UCpaso Muestra mensaje “Error al intentar eliminar el evento".
    	\UCpaso Continua en el paso \ref{CU9_regresar} del \UCref{CU9}.
    \end{UCtrayectoriaA}

	


