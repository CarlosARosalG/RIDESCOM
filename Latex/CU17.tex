\begin{UseCase}{CU17}{Editar datos de las Pruebas}{
		\noindent Esta caso de uso servirá para que el Jefe de Fomento Deportivo pueda editar los datos de las pruebas registradas, con la finalidad de que pueda modificar datos es caso de ser necesario sin problema alguno.
		Para ello el Jefe de Fomento Deportivo dará click en el botón \IUbutton{ Editar } ubicado en la  parte derecha de la tabla de Pruebas dentro de la \IUref{}{Pantalla de Pruebas \ref{VistaPruebas}}.	
	} \label{CU17_evento}
x
		\UCitem{Versión}{0.1}
		\UCitem{Autor}{Rosales González Carlos Andrés}
		\UCitem{Supervisa}{Mendoza García Bruno Alejandro}
		\UCitem{Actor}{Jefe de Fomento Deportivo}
		\UCitem{Propósito}{Editar datos de pruebas Interpolitécnicos Deportivos.}
        \UCitem{Precondiciones}{
        \begin{itemize}
            \item Iniciar sesión.
            \item Tener una prueba registrada.
            \item Seleccionar una prueba para editar.
        \end{itemize}}
        \UCitem{Postcondiciones}{Se muestra la pantalla Editar Pruebas}
		\UCitem{Entradas}{
        \begin{itemize}
        	\item Nombre de la prueba 
        	\item Tipo de prueba 
        	\item Deporte al que pertenece
        \end{itemize}}
		\UCitem{Origen}{Pantalla, Teclado}
		\UCitem{Salidas}{
		\begin{itemize}
		    \item Nombre de la prueba 
		    \item Tipo de prueba 
		    \item Deporte al que pertenece
		\end{itemize}}
		\UCitem{Destino}{Pantalla Editar Pruebas}
		\UCitem{Errores}{
        	\begin{itemize}
        	    \item Los campos están vacíos.
            \end{itemize}
       }
		\UCitem{Observaciones}{}
		\end{UseCase}
	
    \begin{UCtrayectoria}{Principal}
    \UCpaso[\UCactor] Oprime el botón \IUbutton{ Editar } que esta en la \IUref{}{Pantalla principal \ref{pruebas}}.
    \UCpaso Muestra la \IUref{}{Pantalla Registrar un Evento Interpolitécnico Deportivo \ref{editarpruebas}}.
    \UCpaso[\UCactor] Llena los campos solicitados. \label{CU17_regresar}
    \UCpaso[\UCactor] Presiona el botón \IUbutton{ Registrar }.
    \UCpaso Comprueba que los campos no estén vacíos. \Trayref{A}
    \UCpaso Obtiene los valores ingresados
    \UCpaso Válida campos. \Trayref{B}
    \UCpaso Muestra mensaje de confirmación de registro.
    \UCpaso Muestra la \IUref{}{Pantalla principal \ref{VistaPruebas}}.
    \end{UCtrayectoria}
    
    \begin{UCtrayectoriaA}{A}{Campo(s) vacios}
    	\UCpaso Muestra mensaje “Campos Necesario".
    	\UCpaso Continua en el paso \ref{CU17_regresar} del \UCref{CU17}.
    \end{UCtrayectoriaA}

	\begin{UCtrayectoriaA}{B}{Válida campos}
		\UCpaso Muestra el mensaje "Datos incorrectos".
   		\UCpaso Continua en el paso \ref{CU17_regresar} del \UCref{CU17}.
	\end{UCtrayectoriaA}

	


