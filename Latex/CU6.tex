\begin{UseCase}{CU6}{Consulta Calendario de Eventos}{
		\noindent Esta caso de uso servirá para que el actor pueda consultar en la página principal los eventos que han sido registrados y están disponibles para inscribirse. 
		Para poder visualizar la tabla que muestra estos datos, el actor dará click en el botón \IUbutton{ Calendario } ubicados en la \IUref{}{Pantalla de Principal \ref{principalalum}}.
        
	} \label{CU6_evento}
		\UCitem{Versión}{0.1}
		\UCitem{Autor}{Rosales González Carlos Andrés}
		\UCitem{Supervisa}{Mendoza García Bruno Alejandro}
		\UCitem{Actor}{Jefe de Fomento Deportivo, Coordinador de Unidad Académica, Alumno}
		\UCitem{Propósito}{Consultar los eventos disponibles.}
        \UCitem{Precondiciones}{
        \begin{itemize}
            \item Iniciar sesión.
        \end{itemize}}
        \UCitem{Postcondiciones}{Se muestra la pantalla principal}
		\UCitem{Entradas}{
        \begin{itemize}
        	\item Ciclo escolar
        	\item Nombre Eventos
        	\item Deporte
        \end{itemize}}
		\UCitem{Origen}{Pantalla, Teclado}
		\UCitem{Salidas}{
		\begin{itemize}
		    \item Eventos registrados
		\end{itemize}}
		\UCitem{Destino}{Pantalla Principal}
		\UCitem{Errores}{
        	\begin{itemize}
        	    \item No hay eventos registrados.
            \end{itemize}
       }
		\UCitem{Observaciones}{}
		\end{UseCase}
	\pagebreak
	
    \begin{UCtrayectoria}{Principal}
    \UCpaso[\UCactor] Ingresa a la página RIDESCOM.
    \UCpaso Muestra la pantalla \IUref{}{Pantalla de Inicio de Sesión \ref{principalalum}}.
    \UCpaso[\UCactor] Oprime el botón \IUbutton{ Calendario } que esta en la \IUref{}{Pantalla de Principal \ref{principalJFD}}.
    \UCpaso Muestra la \IUref{}{Pantalla de Inicio de Sesión \ref{principalJFD}} en la sección Calendario de Eventos. \Trayref{A} \label{CU6_regresar}
	\UCpaso[\UCactor] Consulta los eventos. 
    \end{UCtrayectoria}

	\begin{UCtrayectoriaA}{A}{No existen registros}
		\UCpaso Muestra mensaje “No existen registros.".
		\UCpaso Continua en el paso \ref{CU6_regresar} del \UCref{CU6}.
	\end{UCtrayectoriaA}
    

	


