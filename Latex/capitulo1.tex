\chapter{Introducción}
	%=========================================================
	%                                                         Introduccion
	%=========================================================
	
	\noindent Este documento presenta la aplicación web RIDESCOM, propuesta por estudiantes de la Escuela Superior de Cómputo (ESCOM) y dirigida para la misma institución, donde se plantean aspectos importantes a considerar en torno a la aplicación como puede ser la problemática que lo origina, las actividades que lo origina, las soluciones que se proponen, el diseño, la investigación realizada acerca de otras aplicaciones similares, los requerimientos necesarios para su desarrollo, etcétera. Pues son todos estos factores importantes para que la aplicación funcione correctamente y sea de apoyo a la solución del problema que se presenta. Es en este apartado donde se explican las razones de ser el proyecto RIDESCOM, se describe el análisis que se lleva a cabo para posteriormente encontrar acciones y requerimientos, así como tecnologías necesarias para la creación de la aplicación. \\
	Siendo así, se cuenta con un apartado de justificación, en el cual se exponen las razones y los diferentes problemas por los cuales RIDESCOM surge como medida para solucionar dichos problemas.\\
	Se detalla aquí la razón de nuestro proyecto, los diferentes caminos que podemos seguir para conseguir resultados favorables ante las problemáticas y lo que se espera tener cuando la aplicación llegue al usuario final. Es bien sabido que debemos enfocarnos en ellos, y en cómo tratar con los problemas para así encontrar la mejor solución, con los mejores beneficios y mayores resultados.\\
	Se establece un marco teórico en donde se describe el entorno en el cual se desarrolla el proyecto, el público al que va dirigida y las aplicaciones existentes.\\
	Además se plantean las ideas y los problemas que orillan a la creación de RIDESCOM para intentar solventarlos. Por otro lado se tiene un análisis sobre las diferentes tecnologías y plataformas computacionales que nos ayudarán a realizar el proyecto, destacando la importancia de éstos y las consistencias en la aplicación y en los objetivos de la última. Por último, se enlistan las diferentes palabras y términos que a lo largo del documento y en la propia aplicación se utilizan, así como una descripción de las mismas, con el objetivo de contextualizar al lector y comprender mejor la aplicación, su estructura, lo que  realiza y como interactua con el usuario final.\\
	Se muestra principalmente el trasfondo de la aplicación y las herramientas que se utilizan para su desarrollo  presentando las aplicaciones ya existentes que realizan tareas similares al proyecto RIDESCOM, su funcionamiento, al público que se dirigen y los propósitos que se consideran realizando comparativas para encontrar aquellos puntos diferenciales entre una aplicación y otra, para así implementar de la mejor manera en la aplicación las características con mayor importancia y que nos ayudarán a lograr los objetivos
	que tiene el proyecto.\\
	Se cuenta con un capítulo dedicado a la aplicación propiamente dicha, en donde se describe la propuesta concreta de la aplicación, las acciones que ésta puede realizar, los usuarios que realizan una interacción con ella, así como las características y herramientas que en ella se contemplan para su funcionamiento mencionando los objetivos general y específicos que se pretenden alcanzar con la aplicación, además en este capítulo se tiene un primer gran acercamiento con el sistema, pues es aquí donde se analizan las acciones que se requieren y se empieza a descubrir y organizar los diferentes requerimientos funcionales y no funcionales que harán que RIDESCOM funcione apropiadamente, logre cumplir sus objetivos y así intentar solventar los problemas que propiciaron su existencia.\\
	Finalmente se cuenta con un apartado dedicado al análisis, diseño y desarrollo de la aplicación. En él se muestra el trabajo que se ha realizado desde que nació la idea hasta el presente día. Se detallan las diversas tareas realizadas para la comprensión del problema, las posibles soluciones, las aplicaciones similares, los prototipos de práctica previos al desarrollo de la aplicación. Aquí se concentran en forma de iteraciones toda acción que se realizó con fin de entender el propósito del proyecto, los diferentes conceptos referentes al análisis, diseño y desarrollo de RIDESCOM. Se muestran así, los avances que se tiene hasta hoy de la aplicación, a los resultados que se ha llegado a lo largo de estos meses de trabajo, los cambios y los problemas que en él se han tenido, así como los logros que se han encontrado. Por último, se dedica una sección en este apartado para anunciar todo aquello que falta realizar y que se implementará en un futuro para que RIDESCOM se realice completa y exitosamente, y con ella los objetivos planteados y a su vez solucionar aquellos problemas que en su momento fueron quien dieron pauta a la creación de esta aplicación.\\
	Así como se explicó, en este documento nos encontramos con un proyecto que pretende servir y apoyar en ciertos aspectos de la Escuela Superior de Cómputo, así como la forma en que se estructura y las maneras en que se fue desarrollando.
	
	