\chapter{Introducción}
	%=========================================================
	%                                                         Introduccion
	%=========================================================
	\section{Introducci\'on}
	
	\noindent Se presentará la aplicación web RIDESCOM, propuesta por estudiantes de la Escuela Superior de Cómputo (ESCOM) y dirigida para la misma institución la cuál ayudará a la gestion y administración de estudiantes deportistas que desean participar a los eventos deportivos que el IPN realiza cada ciclo escolar.// 
	
	A lo largo del mismo se plantean aspectos importantes a considerar en torno a la aplicación, como puede ser la problemática que lo origina, las actividades que lo origina, las soluciones que se proponen, el diseño, la investigación realizada acerca de otras aplicaciones similares, los requerimientos necesarios para su desarrollo, etcétera. Pues son todos estos factores importantes para que la aplicación funcione correctamente y sea de apoyo a la solución del problema que más adelante se describirán detalladamente. Es en este apartado donde se explican las razones de ser el proyecto RIDESCOM, se describe el análisis que se lleva a cabo para posteriormente encontrar acciones y requerimientos, así como tecnologías necesarias para la creación de la aplicación. 
	Siendo así, se cuenta primeramente con un apartado de justificación, en el cual se exponen las razones y los diferentes problemas por los cuales RIDESCOM surge como medida para solucionar dichos problemas.\\
	Se detalla aquí la razón de nuestro proyecto, los diferentes caminos que podemos seguir para conseguir resultados favorables ante las problemáticas y los resultados que se espera tener cuando la aplicación llegue al usuario final. Es bien sabido que debemos enfocarnos en ellos, y en cómo tratar con los problemas para así encontrar la mejor solución, con los mejores beneficios y mayores resultados.\\
	Se establece un marco teórico en donde se describe el entorno en el cual se desarrolla el proyecto, el público al que va dirigida y las aplicaciones.\\
	Además se plantean las ideas y los problemas mismos que orillan a la creación de RIDESCOM para intentar solventarlos. Por otro lado se tiene también un análisis sobre las diferentes tecnologías y plataformas computacionales que nos ayudarán a realizar el proyecto, destacando la importancia de éstos y las consistencias en la aplicación y en los objetivos de la última. Por último, en este apartado se enlistan las diferentes palabras y términos que a lo largo del documento y en la propia aplicación se utilizan, así como una descripción de los mismos, con el objetivo de contextualizar al lector y comprender mejor la aplicación, su estructura, lo que ésta realiza y la interacción con el usuario final.\\
	En esta sección se muestra principalmente el trasfondo de la aplicación y las herramientas que se utilizan para su desarrollo. Se presentan las aplicaciones ya existentes que realizan tarea similares al proyecto RIDESCOM, el funcionamiento de las mismas, el público al que se dirigen y los propósitos que éstas consideran. Se realizan comparativas para encontrar aquellos puntos diferenciales en una aplicación y otra, y así implementar de la mejor manera en la aplicación las características con mayor importancia y que nos ayudarán a lograr los objetivos que más adelante se describirán.\\
	Se cuenta también con un capítulo dedicado a la aplicación propiamente dicha, en donde se describe la propuesta concreta de la aplicación, las acciones que ésta puede realizar, los usuarios que realizan una interacción con ella, así como las características y herramientas que en ella se contemplan para su funcionamiento. Se enlistan los objetivos general y específicos que se pretenden alcanzar con la aplicación, además es en este capítulo se tiene un primer gran acercamiento con el sistema, pues es aquí donde se analizan las acciones que se requieren y se empieza a descubrir y organizar los diferentes requerimientos funcionales y no funcionales que harán que RIDESCOM funcione apropiadamente, logre cumplir sus objetivos y así intentar solventar los problemas que propiciaron su existencia.\\
	Finalmente se cuenta con un apartado dedicado al análisis, diseño y desarrollo de la aplicación. En él se muestra el trabajo que se ha realizado desde que nació la idea hasta el presente día. Se detallan las diversas tareas realizadas para la comprensión del problema, las posibles soluciones, las aplicaciones similares, los prototipos de práctica previos al desarrollo de la aplicación. Aquí se concentran en forma de iteraciones toda acción que se realizó con fin de entender el propósito del proyecto, los diferentes conceptos referentes al análisis, diseño y desarrollo de RIDESCOM. Se muestran así, los avances que se tiene hasta hoy de la aplicación, los resultados a los que se ha llegado a lo largo de estos meses de trabajo, los cambios y los problemas que en él se han tenido, así como los logros que se han encontrado. Por último, se dedica una sección en este apartado para anunciar todo aquello que falta realizar y que se implementará en un futuro para que RIDESCOM se realice completa y exitosamente, y con ella los objetivos planteados, y a su vez solucionar aquellos problemas que en su momento fueron quien dieron pauta a la creación de la aplicación.\\
	Así como se explicó, en este documento nos encontramos con un proyecto que pretende servir y apoyar en ciertos aspectos de la Escuela Superior de Cómputo, así como la forma en que se estructura y las maneras en que se fue desarrollando.
	
	%=========================================================
	%                                                         Justificacion
	%=========================================================
	\section{Justificaci\'on}
	\noindent En este apartado se describen las razones y los  problemas por los cuales la aplicación RIDESCOM surge. Se detalla el “por qué” del proyecto, los diferentes caminos que podemos seguir para conseguir resultados favorables ante las problemáticas y los resultado que se espera tener cuando la aplicación llegue al usuario final. Así bien, a continuación se presenta la problemática que da origen a la aplicación.\\
	La aplicación web RIDESCOM está dirigida principalmente a los alumnos de la ESCOM que practiquen algún deporte impartido por la misma, aunque son los administrativos o responsables del área de departamento de fomento deportivo parte importante de la misma. Es importante establecer que, a pesar que la aplicación podrá ser utilizada por la comunidad en general,  el entorno en el que el usuario final se desarrolla es de suma importancia para la correcta comprensión del problema y de la solución propuesta con RIDESCOM.\\ 
	Así bien, debemos conocer y familiarizarnos con el entorno de la aplicación y los usuarios a los que está destinada. \\
	El Instituto Politécnico Nacional (IPN) es la institución educativa rectora de la educación tecnológica pública en México en los niveles medio superior, superior y posgrado. Tiene como misión formar integralmente capital humano capaz de ejercer el liderazgo en los ámbitos de su competencia, con una visión global, para contribuir al desarrollo social y económico de México. El Instituto se visualiza como una institución de vanguardia incluyente, transparente y eficiente que contribuye al desarrollo global, a través de sus funciones sustantivas, con calidad ética y compromiso social. A lo largo de su historia, el Politécnico se ha caracterizado por ser una Institución que ha evolucionado de acuerdo a las necesidades y realidades del país, reflejando en su imagen sus orígenes y razón de ser, lo que permite su fácil identificación por las personas y llegando a ser coloquialmente como “el Politécnico” o “el Poli”. \cite{hist} \\
	Es el IPN el alma mater de diferentes instituciones y escuelas públicas en México, tal es el caso de la Escuela Superior de Cómputo, escuela donde se procura que la formación de los estudiantes se integral, pues no solo imparten materias referentes a la formación orientada a sus carrera impartida (ingeniería en Sistemas Computacionales), sino contempla diferentes materias enfocadas a desarrollar diferentes aspectos y habilidades que los alumnos pueden poseer, proponiendo además, la posibilidad de participar en clubes y actividades deportivas y culturales. Es claro que la ESCOM se preocupa por lograr en sus alumnos una educación integral y de calidad. Sin embargo, ésto se ve opacado en numerosas ocasiones, pues a causa de la desorganización o mala comunicación entre los integrantes de la comunidad de la ESCOM, no se cumplen completamente el tener esta educación integral de la que se habla, siendo esto un problema. Pues en la ESCOM, además, la población tiende a ser individualista y aislada, provocando así barreras de comunicación y progreso. Dentro del plantel, las diferentes maneras de difusión de información pueden no ser las más óptimas, pues no se alcanza a distribuir de manera correcta a todos los integrantes de la comunidad, por poner un ejemplo, para la participación o inscripción a actividades deportivas, así como el participar en interpolitécnicos de la misma referente a la inscripción y el proceso que todo este conlleva y de esta manera aplicar soluciones para que no se presente más.\\
	Es por ello que se ha idealizado una solución que permita, entre otras cosas, compartir y conocer, así como agilizar los procesos de inscripción de inerpoliécnicos en la ESCOM.\\
	Así, queda claro que la aplicación RIDESCOM pretende informar y agilizar el proceso para inscribir un interpolitécnico, además de estar dirigida a quienes en primer plano sufren del problema anteriormente descrito, y son estos los propios alumnos del plantel.
	
	
	%=========================================================
	%                                                         Problematica
	%=========================================================
	\section{Problem\'atica}
	\noindent La educación es uno de los factores más importantes para el avance y progreso de las personas y sociedades. Además de proveer conocimientos, la educación enriquece la cultura y los valores. La educación es necesaria en todos los sentidos.\\
	La actividad deportiva dentro de las escuelas juega un factor importante dentro de la misma, sin embargo, en latinoamérica se presenta un alto índice de obesidad en los niños y jóvenes.  \cite{problemas}  Ahora bien, en México y específicamente la educación superior, se tiene participación de la comunidad estudiantil pero no es la gran parte con la que se cuenta. 
	(Poner referencia de escuelas o caso de estudio que se hayan realizado para justificar lo anterior y extender más la introduccion de la problemática.)\\
	Nuestro caso de estudio se enfoca en la ESCOM, para ello se realizó una entrevista con el Coordinador de Fomento Deportivo de ESCOM y con el Jefe del Departamento de Fomento Deportivo, con la finalidad de que se nos proporcionará datos de particpación por parte de los estudiantes, a la vez mencionar por los cuales se cree que no exita mayor participación en estos. Uno de los problemas que se tiene es el proceso actual para la inscripción de un interpolitécnico, este suele un poco tardado y fastidioso en cierto punto para el alumno, ya que este tiene que trámitar una constancia de estudios para que la pueda presentar al momento de ir a solicitar su inscripción a un evento interpolitécnico con el coordinador de la Unidad Académica.\\
	Otro de los problemas que tinene actualmente es el proceso de verificación del alumno (su estatus de inscripción), este punto es un requisito para poder participar,si el alumno no esta inscrito en el periodo actual no podrá participar en algun evento interpolitécnico. Sin embargo se mencionó que se llega a presentar el caso de que un alumno participe aun sin cumplir lo antes mencionado.
	Por último se mencionó que no hay un control en las personas que se registran para participar, ya que se a detectado la participación de personas ajenas a la institución. 
	
	%=========================================================
	%                                                         Problematica
	%=========================================================
	\section{Objetivos}
	\subsection{Objetivo General}
	\noindent Desarrollar una aplicación web de apoyo para el Departamento Deportivo de la Escuela Superior de Cómputo (ESCOM) que permita la inscripción de interpolitécnicos, visualizar los eventos próximos y a su vez sea un espacio de información y difusión.
	\subsection{Objetivos Especificos}
	\begin{itemize}
		\item Implementar un mecanismo de validación del estatus académico del alumno. 
		\item Implementar un módulo para la generación de la cédula de inscripción con base en el formato oficial de interpolitécnicos deportivos. 
		\item Implementar un módulo de comunicación con la red social (Facebook). 
		\item Implementar un módulo de consulta de resultados de competencias.
	\end{itemize}