\begin{UseCase}{CU15}{Consulta Pruebas}{
		\noindent Esta caso de uso servirá para que el Jefe de Fomento Deportivo pueda consultar las pruebas registradas.
		Para ello el Jefe de Fomento Deportivo dará click en el botón \IUbutton{ Pruebas } ubicado en la  parte superior de la \IUref{}{Pantalla principal \ref{principalJFD}}.	
	} \label{CU15_evento}

		\UCitem{Versión}{0.1}
		\UCitem{Autor}{Rosales González Carlos Andrés}
		\UCitem{Supervisa}{Mendoza García Bruno Alejandro}
		\UCitem{Actor}{Jefe de Fomento Deportivo}
		\UCitem{Propósito}{Consultar las pruebas registradas.}
        \UCitem{Precondiciones}{
        \begin{itemize}
            \item Iniciar sesión.
            \item Tener registrados datos.	
        \end{itemize}}
        \UCitem{Postcondiciones}{Se muestra la pantalla principal}
		\UCitem{Entradas}{
        \begin{itemize}
        	\item ID
        	\item Prueba
        	\item Tipo de Prueba
        	\item Actividad Deportiva
        \end{itemize}}
		\UCitem{Origen}{Pantalla, Teclado}
		\UCitem{Salidas}{
		\begin{itemize}
		    \item Pruebas de las actividades deportivas.
		    \item No hay resultados registrados.
		\end{itemize}}
		\UCitem{Destino}{Principal Jefe Fomento Deportivo}
		\UCitem{Errores}{
        	\begin{itemize}
			    \item No hay coordinadores registrados.
            \end{itemize}
       }
		\UCitem{Observaciones}{}
		\end{UseCase}
	
    \begin{UCtrayectoria}{Principal}
    \UCpaso[\UCactor] Ingresa a la \IUref{}{Pantalla Principal \ref{principalJFD}}.
    \UCpaso Muestra la \IUref{}{Pantalla Principal \ref{principalJFD}}.
    \UCpaso[\UCactor] Da click en el botón \IUbutton{ Pruebas }.
    \UCpaso Muestra la \IUref{}{Pantalla Principal \ref{pruebas}}. \label{CU15_regresar} \Trayref{A} \Trayref{B}.
    \UCpaso Muestra la tabla de Pruebas.
    \end{UCtrayectoria}
    
    \begin{UCtrayectoriaA}{A}{Filtra resultados}
    	\UCpaso[\UCactor] Selecciona el dato por el cual quiere filtrar en el botón \IUbutton{ Deporte } y/o el botón \IUbutton{ Tipo de Prueba }.
    	\UCpaso Muestra los datos registrados. \Trayref{B}
    \end{UCtrayectoriaA}

	\begin{UCtrayectoriaA}{B}{No hay registros}
		\UCpaso Muestra mensaje “No hay pruebas registradas".
		\UCpaso Continua en el paso \ref{CU15_regresar} del \UCref{CU15}.
	\end{UCtrayectoriaA}


	


