\chapter{Desarrollo}
	
%	\noindent Así, con lo implementado de la aplicación hasta el momento y tomando en cuenta los problemas que surgieron al momento de estar analizando, diseñando y codficando la aplicación, se ha realizado un análisis sobre los resultados obtenidos hasta el momento, y hemos comparado éstos últimos con lo esperado y ya diseñado de la aplicación, pues como se ha explicado con anterioridad, RIDESCOM se ha divido en módulos para obtener mejores resultados en el tiempo requerido, siendo estos módulos: Mecanismo de Validación de estatus académico, Interfaces, Módulo de formulario, Módulo de difusión de eventos, Módulo de comunicación con redes sociales, Módulo de creación de cédula de inscripción a un evento interpolitécnico deportivo, Módulo de consulta de resultados. Bien, de los ya mencionados y su desarrollo, así como sus respectivas implementaciones se plantea el trabajo a realizar y a entrgar a futuro. A continuación se detalla el avance planeado en cada uno de los módulos mencionados, mismos que juntos completan la aplicación RIDESCOM.
%	
%	
%	\section{Mecanismo de Validación de estatus académico}	
%	 El módulo de Mecanismo de Validación de estatus académico se ha desarrollado hasta el momento un prototipo, el cual muestra una tabla en la que se puede verificar si el alumno esta inscrito o no. Sin embargo, no cuenta con el diseño final ni con las funciones completas.
%	
%	\section{Interfaces}
%	Las interfaces se han desarrollado hasta el momento prototipos, el cual muestra de manera genreal, la estructura y el contenido de las vistas que conforman la aplicación web RIDESCOM. Sin embargo, conforme se va análisando cada módulo pueden cambiar como ya pasado. 
%	
%	\section{Módulo de formulario}	
%	El módulo de formulario originalmente estaba destinado para desarrollar los formularios de Registro, Inscripción a un evento interpolitécnico deportivo, sin embargo como ya se mencionó, se desarrolló un crawler que se conecta con el SAES dando una mejor solución a la problematica y a su vez omitiendo de cierta manera algunas vistas.
%	\pagebreak
%	
%	\section{Módulo de difusión de eventos}
%	La finalidad de éste módulo es que el coordinador de la unidad académica pueda postear en la red social de Facebook de la escuela, un evento que ya haya sido registrado previamente para que de esta manera pueda llegar a la mayor población posible y así, tratar conseguir más participación de la comunidad.
%	
%	
%	\section{Módulo de comunicación con redes sociales}	
%	Esté módulo esta ligado con el Módulo de difusión de eventos ya que para poder hacer la difusión de los eventos debe existir una conexión entre RIDESCOM y Facebook para que se pueda concluir con la difusión.
%	
%	
%	\section{Módulo de creación de cédula de inscripción a un evento interpolitécnico deportivo}	
%	Dentro de éste módulo se generará el archivo que servirá como evidencia para los coordinadores de las unidades académicas, a su vez se hará uso del Mecanismo de Validación de estatus académico.
%	
%	\section{Módulo de consulta de resultados}	
%	Finalmente se plantea el Módulo de consulta de resultados, que servirá para informar a los participantes y la comunidad en general los resultados de los compañeros que participaron en los eventos interpolitécnicos deportivos. 
 
	