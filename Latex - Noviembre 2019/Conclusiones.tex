\chapter*{Conclusiones}
\phantomsection
\addcontentsline{toc}{chapter}{Conclusiones}
	\noindent 
	Durante el desarrollo de Trabajo Terminal 1 y Trabajo Terminal 2, nos pudimos dar cuenta
	de todos los aspectos que deben de tomarse en cuenta para el desarrollo de un proyecto. \\
	Comenzando por el análisis y toma de requerimientos para tener bases sólidas a lo largo del proyecto, una constante comunicación con el equipo de trabajo y el cliente para obtener el mejor resultado posible. \\
	Cabe destacar que dentro del análisis se debe tener bastante investigación acerca de las herramientas que se emplearán, ya que esta puede beneficiar o atrasar el desarrollo del proyecto, a su vez se debe delimitar el alcance del proyecto, tomando como base los requisitos del usuario y sus necesidades para así, realizar un proyecto que ayude al problema que se quiere atacar. \\
	Otro punto importante que nos percatamos es considerar las habilidades y conocimientos del equipo de trabajo, ya que si el equipo no tiene las habilidades necesarias se ve reflejado en el proyecto. \\
	Ahora bien, el proyecto “Prototipo para el registro a eventos interpolitécnicos deportivos RIDESCOM”, cumple con los objetivos propuestos cubriendo las problemáticas que se tenían y sobre todo con esta propuesta, se sigue el Reglamento estipulado. Haciendo énfasis en la problemática principal, la participación de personas ajenas al Instituto Politécnico Nacional. \\
	Sin embargo cabe mencionar que a lo largo del desarrollo se presentaron complicaciones que van desde el aprender a usar un nuevo Frame que ayude a mejorar el desarrollo de la aplicación hasta la implementación del Crawler. No obstante, se aprendió a ser autodidactas, el adentrarse y adaptarnos a nuevas tecnologías de desarrollo con la finalidad de obtener el mejor resultado posible para el proyecto, con todo esto crecemos profesionalmente brindando mejor opinión, desarrollo y trabajo final. Junto con el apoyo y orientación de nuestros directores del Trabajo Terminal. \\
	
	