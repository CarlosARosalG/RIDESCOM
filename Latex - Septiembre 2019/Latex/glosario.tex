\chapter{Glosario}
	Para el mejor entendimiento de este documento se enlistaron las diferentes palabras y términos que a lo largo del documento y en la propia aplicación se utilizan, y una descripción de los mismos, con el objetivo de contextualizar al lector y comprender mejor la aplicación, su estructura, lo que esta realiza y la interacción que tiene con el usuario final.
	\begin{itemize}
		\item \textbf{Actividad deportiva:} Tiempos, espacios y equipos deportivos ofrecidos por parte de la ESCOM hacia los alumnos para complementar su formación.
		\item \textbf{Alumno:} Persona que cuenta con un número de boleta y está inscrito en la ESCOM. Además debe estar registrado en el sistema. 
		\item \textbf{Aplicación Web:} Aplicación que los usuarios pueden utilizar accediendo a un servidor web a través de Internet o de una intranet mediante un navegador.
		\item \textbf{Área:}
		\item \textbf{Base de datos:} Conjunto de datos pertenecientes a un mismo contexto y almacenados sistemáticamente para su posterior uso. 
		\item \textbf{Boleta:} Identificador único de cada alumno dentro del IPN, a usar dentro del sistema, proporcionado por el IPN a los alumnos inscritos. 
		\item \textbf{Cédula de inscripción:} Documento identificativo utilizado para la inscripción de personas para eventos deportivos.
		\item \textbf{Contraseña:} Clave de acceso conformada por caracteres alfanuméricos asociada a una boleta o número de empleado.
		\item \textbf{Entorno de desarrollo integrado / IDE:} Aplicación informática que proporciona servicios integrales para facilitarle al desarrollador o programador el desarrollo de software. 
		\item \textbf{Escuela Superior de Cómputo / ESCOM:}  Institución pública mexicana de educación superior perteneciente al Instituto Politécnico Nacional. 
		\item \textbf{Iniciar sesión:} Sección del sistema que autentica al usuario mediante una boleta o un número de empleado y una contraseña, permitiéndonos identificar su tipo (alumno, profesor) brindándole acceso a su perfil. 
		\item \textbf{Instituto Politécnico Nacional:} Institución pública mexicana de investigación y educación en niveles medio superior, superior y posgrado. 
		\item \textbf{Interpolitécnico:} Evento competitivo que es desarrollado en el Instituto Politécnico Nacional (IPN) con la finalidad de fortalecer el sistema de competición integral de los alumnos en alguna disciplinas.
		\item \textbf{Número de empleado:} Identificador único de cada profesor dentro del IPN, a usar dentro del sistema, proporcionado por el IPN a los profesores contratados. 
		\item \textbf{Requisito funcional:} Función del sistema de software o sus componentes. Función es descrita como un conjunto de entradas, comportamientos y salidas. 
		\item \textbf{Requisito no funcional:} Requisito que sabe bien y especifica criterios que pueden usarse para juzgar la operación de un sistema en lugar de sus comportamientos específicos, ya que estos corresponden a los requisitos funcionales. 
		\item \textbf{Servicio Web:} Tecnología que utiliza un conjunto de protocolos y estándares que sirven para intercambiar datos entre aplicaciones.
		\item \textbf{Servidor:} Aplicación software en ejecución capaz de atender las peticiones de un cliente y devolverle una respuesta en concordancia.
		\item \textbf{Sistema gestor de base de datos / SGBD:} Conjunto de programas que permiten el almacenamiento, modificación y extracción de la información en una base de datos, además de proporcionar herramientas para a~nadir, borrar, modificar y analizar los datos.
		\item \textbf{Software:} Conjunto de programas y rutinas que permiten a la computadora realizar determinadas tareas.
		\item \textbf{Unidad de aprendizaje:} Curso impartido en la ESCOM y que tiene la intención educativa para que se apliquen y se adquieran conocimientos con el n de que los alumnos desarrollen competencias como el pensamiento estratégico, el pensamiento creativo, trabajo colaborativo, trabajo participativo, ética, manejo de conflictos, responsabilidad social, comunicación asertiva, actitud emprendedora.
		\item \textbf{Usuario:} Conjunto de permisos y de recursos (o dispositivos) a los cuales se tiene acceso. Es decir, un usuario puede ser tanto una persona como una máquina, un programa, etc.
	\end{itemize}

