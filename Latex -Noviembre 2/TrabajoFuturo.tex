\chapter*{Trabajo a Futuro}
\phantomsection
\addcontentsline{toc}{chapter}{Trabajo a Futuro}
	\noindent Tomando como base el trabajo realizado y obtenido a lo largo de Trabajo Terminal 1 y Trabajo Terminal 2 se puede tomar como base lo realizado, sin embargo, creemos que se pueden agregar otros módulos que pueden complementar el proyecto, tales módulos son: módulo de estadísticas de participación de las escuelas y/o deportes mostrando la participación estudiantil varonil y femenil, implementar la difusión de eventos con la red social Faceboook. Está última, estaba contemplada dentro de los objetivos de Trabajo Terminal 1 y Trabajo Terminal 2 sin embargo, no se pudo completar dado que constantemente los términos y condiciones de los pluggin de Facebook cambian y al final se solicitaba la comprobación del usuario que desarrollaba la app y la verificación del negocio. \\
	
	\noindent Una vez tomado en cuenta los puntos y observaciones antes mencionados se podría implementar en otras Unidades Académicas a nivel superior, tomando en cuenta el análisis de impacto del proyecto, así como posibles módulos que puedan surgir y complementar el proyecto finalmente, incluir el proyecto en todas las unidades académicas del Instituto Politécnico Nacional. El origen y motivo principal de este proyecto es el que exista una herramienta que ayude al departamento de Fomento Deportivo la administración y control de los alumnos participantes de los eventos interpolitécnicos deportivos y así lograr la participación de más alumnos en estos y sobre todo, seguir el Reglamento estipulado. \\
	
	