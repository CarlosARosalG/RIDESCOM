\begin{UseCase}{CU21}{Eliminar Deportes}{
		\noindent Esta caso de uso servirá para que el Jefe de Fomento Deportivo pueda eliminar los Deportes registrados para los eventos interpolitécnicos.
		Para ello el Jefe de Fomento Deportivo dará click en el botón \IUbutton{ Eliminar } ubicado en la  parte derecha de la tabla de Pruebas dentro de la \IUref{}{Pantalla de Deportes \ref{deportes}}.	
	} \label{CU21_evento}

		\UCitem{Versión}{0.1}
		\UCitem{Autor}{Rosales González Carlos Andrés}
		\UCitem{Supervisa}{Mendoza García Bruno Alejandro}
		\UCitem{Actor}{Jefe de Fomento Deportivo}
		\UCitem{Propósito}{Eliminar datos de Deportes para los Interpolitécnicos Deportivos.}
        \UCitem{Precondiciones}{
        \begin{itemize}
            \item Iniciar sesión.
        \end{itemize}}
        \UCitem{Postcondiciones}{Se muestra la pantalla de pruebas}
		\UCitem{Entradas}{
        \begin{itemize}
        	\item Iniciar sesión.
        \end{itemize}}
		\UCitem{Origen}{Pantalla, Teclado}
		\UCitem{Salidas}{
		\begin{itemize}
		    \item Iniciar sesión.
		\end{itemize}}
		\UCitem{Destino}{Principal Deprotes}
		\UCitem{Errores}{
        	\begin{itemize}
        	    \item Los campos están vacíos.
            \end{itemize}
       }
		\UCitem{Observaciones}{}
		\end{UseCase}
	
    \begin{UCtrayectoria}{Principal}
    \UCpaso[\UCactor] Ingresa a la \IUref{}{Pantalla de pruebas \ref{deportes}}.
    \UCpaso Muestra la \IUref{}{Pantalla de deportes \ref{deportes}}.
    \UCpaso[\UCactor] Da click en el botón \IUbutton{ Eliminar }. \label{CU21_regresar}.
    \UCpaso Muestra mensaje para confirmar la acción. \Trayref{A}
    \UCpaso Muestra la \IUref{}{Pantalla principal del Jefe de Fomento Deportivo. \ref{deportes}}.
    \end{UCtrayectoria}
    
    \begin{UCtrayectoriaA}{A}{Error al eliminar el evento}
    	\UCpaso Muestra mensaje “Error al intentar eliminar el evento".
    	\UCpaso Continua en el paso \ref{CU21_regresar} del \UCref{CU21}.
    \end{UCtrayectoriaA}

	\begin{UCtrayectoriaA}{B}{Válida campos}
		\UCpaso Muestra el mensaje "Datos incorrectos".
   		\UCpaso Continua en el paso \ref{CU21_regresar} del \UCref{CU21}.
	\end{UCtrayectoriaA}

	


